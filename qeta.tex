\documentclass{article}
\usepackage{qeta}

\begin{document}
\title{The QEta Package}
\author{Ralf Hemmecke}
\maketitle
\begin{abstract}
  The QEta package is a collection of programs written in the
  FriCAS\footnote{FriCAS~1.3.2~\cite{FriCAS}} computer algebra system
  that allow to compute with Dedekind $\eta$-function and related
  $q$-series where $q=e^{2\pi i}$.
  \url{https://en.wikipedia.org/wiki/Dedekind_eta_function}
\end{abstract}

\tableofcontents

%%%%%%%%%%%%%%%%%%%%%%%%%%%%%%%%%%%%%%%%%%%%%%%%%%%%%%%%%%%%%%%%%%%
\section{General Overview}
%%%%%%%%%%%%%%%%%%%%%%%%%%%%%%%%%%%%%%%%%%%%%%%%%%%%%%%%%%%%%%%%%%%

The QEta package contains implementation of the \algo{AB} algorithm
from \cite{Radu:RamanujanKolberg:2015} and the \algoSamba{} algorithm
from \cite{Hemmecke:DancingSambaRamanujan:2018}, in addition it
implements the algorithm from \cite{Hemmecke+Radu:EtaRelations:2018}
to compute all polynomial relations of Dedekind $\eta$-functions of a
certain level.

The underlying theory of the programs is described in the articles
\cite{Radu:RamanujanKolberg:2015},
\cite{Hemmecke:DancingSambaRamanujan:2018}, and
\cite{Hemmecke+Radu:EtaRelations:2018}.

This package requires a version of FriCAS that is compiled from at
least SVN revision 2328, \ie, where \GB{} computations do no longer
require variable names. I fact, the scripts usually use
\begin{verbatim}
)set output linear on
\end{verbatim}
which is coded in the file \PathName{1d.spad} in the
\code{master-hemmecke} branch of a clone of the FriCAS git repository.
\url{https://github.com/hemmecke/fricas/commits/master-hemmecke}
However, this \emph{linear} output form is not absolutely necessary.

%%%%%%%%%%%%%%%%%%%%%%%%%%%%%%%%%%%%%%%%%%%%%%%%%%%%%%%%%%%%%%%%%%%
\section{Overview of the files}
%%%%%%%%%%%%%%%%%%%%%%%%%%%%%%%%%%%%%%%%%%%%%%%%%%%%%%%%%%%%%%%%%%%

The QEta package consists of following parts that are stored in the
respective \PathName{.spad} files. We mark the files that are only
there for historical reasons by a star. They are not really necessary
to compute the relation among $\eta$-functions.

\include{qetaabstracts}

\bibliography{qeta}
\end{document}
