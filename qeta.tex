\documentclass{article}
\usepackage{qeta}
\externaldocument{qetaquotsymb}
\externaldocument{qetaquot}

\begin{document}
\title{The QEta Package}
\author{Ralf Hemmecke}
\maketitle
\begin{abstract}
  The QEta package is a collection of programs written in the
  FriCAS\footnote{FriCAS~1.3.2~\cite{FriCAS}} computer algebra system
  that allow to compute with Dedekind eta-function and related
  $q$-series where $q=e^{2\pi i}$, see
  \url{https://en.wikipedia.org/wiki/Dedekind_eta_function}.
  Furthermore, we provide a number of functions connected to the
  theory of modular functions.
\end{abstract}

\tableofcontents

%%%%%%%%%%%%%%%%%%%%%%%%%%%%%%%%%%%%%%%%%%%%%%%%%%%%%%%%%%%%%%%%%%%
\section{General Overview}
%%%%%%%%%%%%%%%%%%%%%%%%%%%%%%%%%%%%%%%%%%%%%%%%%%%%%%%%%%%%%%%%%%%

The QEta package started with an implementation of the \algo{AB}
algorithm from \cite{Radu:RamanujanKolberg:2015} and the \algoSamba{}
algorithm from \cite{Hemmecke:DancingSambaRamanujan:2018}, in addition
it implements the algorithm from
\cite{Hemmecke+Radu:EtaRelations:2018} to compute all polynomial
relations of Dedekind eta-functions of a certain level.

The underlying theory of the programs is described in the articles
\cite{Radu:RamanujanKolberg:2015},
\cite{Hemmecke:DancingSambaRamanujan:2018}, and
\cite{Hemmecke+Radu:EtaRelations:2018}.

This package requires a version of FriCAS that is compiled from a
commit after end of December, 2020. \ie, where \GB{} computations do
no longer require variable names and the new formatting framework was
introduced.

%%%%%%%%%%%%%%%%%%%%%%%%%%%%%%%%%%%%%%%%%%%%%%%%%%%%%%%%%%%%%%%%%%%
\section{Overview of the files}
%%%%%%%%%%%%%%%%%%%%%%%%%%%%%%%%%%%%%%%%%%%%%%%%%%%%%%%%%%%%%%%%%%%

The QEta package consists of following parts that are stored in the
respective \PathName{.spad} files. We mark the files that are only
there for historical reasons by a star. They are not really necessary
to compute the relation among eta-functions.

\include{qetaabstracts}

%%%%%%%%%%%%%%%%%%%%%%%%%%%%%%%%%%%%%%%%%%%%%%%%%%%%%%%%%%%%%%%%%%%
\section{Theoretical background}
%%%%%%%%%%%%%%%%%%%%%%%%%%%%%%%%%%%%%%%%%%%%%%%%%%%%%%%%%%%%%%%%%%%

%%%%%%%%%%%%%%%%%%%%%%%%%%%%%%%%%%%%%%%%%%%%%%%%%%%%%%%%%%%%%%%%%%%
\subsection{Notation}
%%%%%%%%%%%%%%%%%%%%%%%%%%%%%%%%%%%%%%%%%%%%%%%%%%%%%%%%%%%%%%%%%%%

\begin{gather}
  \defineNotation[GL2+Z]{GL_2^+(\setZ)}
  :=
  \SetDef{  \begin{pmatrix}
    a & b\\
    c & d
  \end{pmatrix}}{a,b,c,d\in\setZ \land ad-bc>0}
  \label{eq:GL2+Z}
\end{gather}

\begin{gather}
  \defineNotation[SL2Z]{\SL2Z} := \SetDef{\gamma\in
    GL_2^+(\setZ)}{\det(\gamma)=1}
  \label{eq:SL2Z}
\end{gather}

Let $N$ be a positive natural number and define a $\Gamma_0(N)$ as a
subgroup of the special linear group $\SL2Z$ as follows.
\begin{gather}
\defineNotation[Gamma0N]{\Gamma_0(N)} := \SetDef{  \begin{pmatrix}
    a & b\\
    c & d
  \end{pmatrix} \in \SL2Z}{c \equiv 0 \pmod{N}}
  \label{eq:Gamma0}
\end{gather}

Let $N$ be a positive natural number and define a $\Gamma_1(N)$ as a
subgroup of the special linear group $\SL2Z$ as follows.
\begin{gather}
\defineNotation[Gamma1N]{\Gamma_1(N)} := \SetDef{  \begin{pmatrix}
    a & b\\
    c & d
  \end{pmatrix} \in \SL2Z}{c\equiv0 \pmod{N}, a\equiv d\equiv1 \pmod{N}}
  \label{eq:Gamma1}
\end{gather}

See, for example,
\url{https://en.wikipedia.org/wiki/Congruence_subgroup}.

As in \cite{Radu:RamanujanKolberg:2015}, let $K(N)$ denote the set of
modular functions for $\Gamma_0(N)$ and $K^\infty(N)$ the set of
modular functions have a (multiple) pole, if any, at infinity only.
Furthermore, let $E(N)$ be the set of eta-quotients that are
modular functions, and let $E^\infty(N) := E(N)\cap K^\infty(N)$.

%%%%%%%%%%%%%%%%%%%%%%%%%%%%%%%%%%%%%%%%%%%%%%%%%%%%%%%%%%%%%%%%%%%
\begin{Definition}\label{def:epsilon}
  Let $\defineNotation[epsilon]{\unityPowerSymbol}: \setC \to \setC$
  be defined by
  \begin{gather*}
    \unityPower{x} := \exp(2\pi i x).
  \end{gather*}
\end{Definition}
%%%%%%%%%%%%%%%%%%%%%%%%%%%%%%%%%%%%%%%%%%%%%%%%%%%%%%%%%%%%%%%%%%%

Let $\defineNotation[H]{\setH}=\SetDef{c\in \setC}{\Im(c)>0}$ denote
the complex upper half-plane.

Let
\begin{gather}\label{eq:eta-expansion}
  \defineNotation[eta]{\eta}: \setH \to \setC, \quad
  \tau \mapsto \unityPower{\frac{\tau}{24}} \prod_{n=1}^{\infty}(1-q^n)
  =
  \unityPower{\frac{\tau}{24}} \eulerFunction{}
\end{gather}
with $\defineNotation[q]{q} = \unityPower{\tau}$
denote the Dedekind eta-function.

Note that $\qPochhammer{a}{q} = \prod_{n=0}^{\infty}(1-aq^n)$ denotes
the $q$-Pochhammer symbol. Thus, we have
\begin{gather*}
  \eta(\tau) = \unityPower{\frac{\tau}{24}} \eulerFunction{}
\end{gather*}

In the following $\defineNotation[N]{N}$ denotes a positive integer
and $1=\delta_1<\delta_2\dots<\delta_{\defineNotation[n]{n}}=N$ the
positive divisors of $N$. Let
$\defineNotation[Delta]{\Delta}:=\Set{\delta_1,\ldots,\delta_n}$. For
convenience, we allow to index $n$-dimensional vectors by the divisors
of $N$, instead of the usual index set $\Set{1,\ldots,n}$.
%
For $\delta\in\Delta$ we consider the functions
\begin{gather*}
  \defineNotation[eta-delta]{\eta_\delta}: \setH \to \setC,\quad \tau
  \mapsto \eta(\delta\tau)
\end{gather*}
None of these functions is identically zero.




%%%%%%%%%%%%%%%%%%%%%%%%%%%%%%%%%%%%%%%%%%%%%%%%%%%%%%%%%%%%%%%%%%%
When we write $\divisorsum{N}$ and $\divisorprod{N}$ we mean the sum
and product over all \emph{positive} divisors of $N$.

%%%%%%%%%%%%%%%%%%%%%%%%%%%%%%%%%%%%%%%%%%%%%%%%%%%%%%%%%%%%%%%%%%%
We define $\defineNotation[RN]{R(N)}$ to be the set of integer tuples
$\defineNotation[r]{r} = (r_{\delta_1}, \ldots,
r_{\delta_n})\in \setZ^n$.
%
For $r\in R(N)$ let
\begin{align}
  \defineNotation[g-r-tau]{g_r(\tau)}
  &:=
    \divisorprod{N} \eta(\delta\tau)^{r_\delta}
    =
    q^{{\sumdelta{r}}/{24}}
      \divisorprod{N} \eulerFunction{\delta}^{r_\delta}
  \label{eq:g_r(tau)}
\end{align}
where we have abbreviated
\begin{gather}
  \defineNotation[r^{}]{\sumdelta{r}} := \divisorsum{N} \delta r_\delta.
\end{gather}




By $\defineNotation[R*N]{R^*(N)}$ we denote the subset of all tuples
$r=(r_\delta)_{\delta\in\Delta}$ of $R(N)$ that fulfil the following
conditions.
\begin{align}
 \divisorsum{N} r_\delta &= 0\label{eq:sum=0}\\
 \divisorsum{N} \delta r_\delta &\equiv 0\pmod{24}\label{eq:sigmainfinity}\\
 \divisorsum{N} (N/\delta)r_\delta &\equiv 0\pmod{24}\label{eq:sigma0}\\
 \sqrt{\divisorprod{N}\delta^{r_\delta}} &\in \setQ\label{eq:productsquare}
\end{align}

Note that $R^*(N)$ is an additive monoid.

%%%%%%%%%%%%%%%%%%%%%%%%%%%%%%%%%%%%%%%%%%%%%%%%%%%%%%%%%%%%%%%%%%%
\subsection{Overview}
%%%%%%%%%%%%%%%%%%%%%%%%%%%%%%%%%%%%%%%%%%%%%%%%%%%%%%%%%%%%%%%%%%%

According to
\cite[Lemma~5.3]{Radu:AlgebraicRelationsInvolvingEtaQuotients:2016},
there are $\eulerphi(\gcd(N/c, c))$ different cusps $\frac{a}{c}$ of
$\Gamma_0(N)$ that correspond to a divisor $c$ of $N$ where
$\defineNotation[phi]{\eulerphi}$ denotes the Euler totient function.

In the following we always deal with quotients of eta-functions of
level $N$ that are modular functions for $\Gamma_0(N)$.

We implement a domain that deals with such quotients and their
$q$-expansion at all cusps ($q=e^{2\pi i \tau})$.

%%%%%%%%%%%%%%%%%%%%%%%%%%%%%%%%%%%%%%%%%%%%%%%%%%%%%%%%%%%%%%%%%%%
\subsection{Transformations of the Dedekind eta-function}
%%%%%%%%%%%%%%%%%%%%%%%%%%%%%%%%%%%%%%%%%%%%%%%%%%%%%%%%%%%%%%%%%%%

%%%%%%%%%%%%%%%%%%%%%%%%%%%%%%%%%%%%%%%%%%%%%%%%%%%%%%%%%%%%%%%%%%%

Let's look at the transformation of the eta-function (see
\cite[Lemma~2.27]{Radu:PhD:2010}).

Let
$\defineNotation[gamma]{\gamma} =
\left(\begin{smallmatrix}a&b\\c&d\end{smallmatrix}\right) \in
\SL2Z$, then
\begin{gather}
\defineNotation[eta]{\eta(\gamma\tau)} =
\eta\left(\frac{a\tau+b}{c\tau+d}\right) =
(c\tau+d)^{1/2}\,\upsilon(\gamma)\,\eta(\tau)
\label{eq:eta-transformation}
\end{gather}
where
$\defineNotation[upsilon-gamma]{\upsilon(\gamma)} :=
\upsilon_\eta(a,b,c,d)$ as defined as in
\cite[Lemma~2.27]{Radu:PhD:2010}.

%%%%%%%%%%%%%%%%%%%%%%%%%%%%%%%%%%%%%%%%%%%%%%%%%%%%%%%%%%%%%%%%%%%
\begin{Lemma}
\label{thm:matix-splitting}
Let $\gamma',\gamma''\in \SL2Z$.
%
Then $\upsilon(\gamma'\gamma'')^2 = \upsilon(\gamma')^2 \upsilon(\gamma'')^2$.
\end{Lemma}
%%%%%%%%%%%%%%%%%%%%%%%%%%%%%%%%%%%%%%%%%%%%%%%%%%%%%%%%%%%%%%%%%%%
\begin{proof}
  Let
  $\gamma'=\left(\begin{smallmatrix}a'&b'\\c'&d'\end{smallmatrix}\right)$
  and
  $\gamma''=\left(\begin{smallmatrix}a''&b''\\c''&d''\end{smallmatrix}\right)$
  be in $\SL2Z$.
  %
  Then
  $\gamma'\gamma'' =
  \left(
    \begin{smallmatrix}
      a'a''+b'c'' & a'b''+b'd''\\
      c'a''+d'c'' & c'b''+d'd''
    \end{smallmatrix}
  \right)$.
  We set $c:=c'a''+d'c''$ and $d:=c'b''+d'd''$.
%
  If $\tau'=\gamma''\tau$, we derive for any $\tau \in \setH$:
  \begin{gather*}
    c' \tau' + d'
    =
      \left(c' \frac{a''\tau + b''}{c''\tau + d''} +d'\right)
    =
      \left(\frac{c'a''\tau + c'b''+ d'c''\tau + d'd''}{c''\tau + d''}\right)
    =
    \left(\frac{c\tau + d}{c''\tau + d''}\right).
  \end{gather*}
%
  \begin{align*}
    (c \tau + d)^{1/2}\,\upsilon(\gamma)\,\eta(\tau)
    &=
    \eta((\gamma'\gamma'')\tau) =
    \eta(\gamma'\tau')
    =
      (c'\tau'+d')^{1/2}\,\upsilon(\gamma')\,\eta(\tau')\\
    &=
      \left(\frac{c\tau + d}{c''\tau + d''}\right)^{1/2} \,(c''\tau+d'')^{1/2}
      \,\upsilon(\gamma')\,\,\upsilon(\gamma'')\eta(\tau).
  \end{align*}
  After squaring both sides and cancelling equal terms, the statement
  of the Lemma can be concluded.
\end{proof}



%%%%%%%%%%%%%%%%%%%%%%%%%%%%%%%%%%%%%%%%%%%%%%%%%%%%%%%%%%%%%%%%%%%
\subsubsection{Transformation of $\eta$ under $GL_2^+(\setZ)$}
\label{sec:eta-transformation}
%%%%%%%%%%%%%%%%%%%%%%%%%%%%%%%%%%%%%%%%%%%%%%%%%%%%%%%%%%%%%%%%%%%

\begin{Lemma}
  \label{thm:W-splitting}
  Let
  $W=\left(\begin{smallmatrix}A&B\\C&D\end{smallmatrix}\right) \in
  GL_2^+(\setZ)$.
  %
  Then for $h=\gcd(A, C)>0$, $a':=\frac{A}{h}$, $c':=\frac{C}{h}$ we
  have $\gcd(a', c')=1$ and thus can find $b'$ and $d'$ such that
  $a'd'-b'c'=1$.
  %
  Let
  \begin{gather*}
    W':=\begin{pmatrix}a'&b'\\c'&d'\end{pmatrix},
    \qquad
    W''
    :=
    \begin{pmatrix}
      h &B d'-D b'\\
      0 & \det(W) / h
    \end{pmatrix}.
  \end{gather*}
  Then $W'\in \SL2Z$ and $W = W' W''$.
\end{Lemma}


Note that for any integer $s$ whe have:
\begin{align}
W = \begin{pmatrix}A&B\\C&D\end{pmatrix}
&=\begin{pmatrix}a'&b'\\c'&d'\end{pmatrix}
  \begin{pmatrix}1& s\\0&1\end{pmatrix}
  \begin{pmatrix}1&-s\\0&1\end{pmatrix}
  \begin{pmatrix}h &B d'- D b'\\0 &\det(W) / h\end{pmatrix}
  \\
&=
  \begin{pmatrix} a'&b' + a' s\\ c'&d' + c' s \end{pmatrix}
  \begin{pmatrix}
    h &B d'- D b'- s \det(W) / h\\
    0 &\det(W) / h\end{pmatrix}.
\end{align}
Therefore, we can choose $b'$ and $d'$ in such a way that
$0 \le B d'-D b' < \det(W)/h$.

Let $\tau' := W''\tau = \frac{h}{\det(W)}(h\tau + B d'- D b')$, then
$\tau'\in\setH$ and
\begin{gather}
\eta(W\tau) =
\eta(W'\tau') =
(c'\tau+d')^{1/2}\,\upsilon_\eta(W')\,\eta(\tau')
\label{eq:eta-W-transformation}
\end{gather}

%%%%%%%%%%%%%%%%%%%%%%%%%%%%%%%%%%%%%%%%%%%%%%%%%%%%%%%%%%%%%%%%%%%
\begingroup
\newcommand{\h}{h}
\begin{Lemma}\label{thm:c*tau+d}
  Let $\gamma := \left(
    \begin{smallmatrix}a&b\\c&d\end{smallmatrix} \right)\in
  \SL2Z$, $\delta>0$, $m>0$, $z\ge0$, and
  \begin{gather*}
    W=
    \begin{pmatrix}
      \delta & z\\
      0      & m
    \end{pmatrix}
    \gamma
    = \begin{pmatrix}
        \delta a + c z & \delta b + d z\\
        c m & d m
      \end{pmatrix}
    \in GL_2^+(\setZ).
  \end{gather*}
  Then $\det(W)=\delta m$.
  %
  If
  $\h:=\gcd(\delta a + c z, cm)$,
  $a':=\frac{\delta a + c z}{\h}$,
  $c':=\frac{c m}{\h}$, and
  $b'$ and $d'$ are such that $a'd'-b'c'=1$ and
  $0 \le (\delta b + d z) d' - d m b' < cm / \h$, \ie,
  \begin{gather*}
    W':=\begin{pmatrix}
      (\delta a + c z) / \h & b'\\
      c m / \h              & d'
    \end{pmatrix},
    \qquad
    W''
    :=
    \begin{pmatrix}
      \h & (\delta b + d z) d' - d m b'\\
      0  & \delta m / \h
    \end{pmatrix}.
  \end{gather*}
  according to the splitting above,
  %
  then with $\tau':=W''\tau$ we have
  $c'\tau'+d' = \frac{\h}{\delta} (c\tau + d)$.
\end{Lemma}
%%%%%%%%%%%%%%%%%%%%%%%%%%%%%%%%%%%%%%%%%%%%%%%%%%%%%%%%%%%%%%%%%%%
\begin{proof}
\begin{align*}
  c'\tau'+d'
  &=
  c'
    \left(
    \frac{\h}{\delta m}(h\tau + (\delta b + d z) d' - d m b')
    \right) + d'\\
  &=
    \frac{c m}{\h}
    \left(
    \frac{\h}{\delta m}(\h \tau + (\delta b + d z) d' - d m b')
    \right) + d'\\
  &=
    \frac{1}{\delta}
    \left(
    \h c \tau + b c \delta d' + c d d' z - c d m b'
    \right) + d'\\
  &=
    \frac{1}{\delta}
    \left(
    \h c \tau + (a d - 1) \delta d' + c d d' z - c d m b'
    + \delta d'
    \right)\\
  &=
    \frac{1}{\delta}
    \left(
    \h c \tau + a d \delta d' + c d d' z - c d m b'
    \right)\\
  &=
    \frac{1}{\delta}
    \left(
    \h c \tau + \h d (a' d' - c' b')
    \right)\\
  &=
    \frac{\h}{\delta} (c\tau + d)
\end{align*}
\end{proof}
\endgroup

%%%%%%%%%%%%%%%%%%%%%%%%%%%%%%%%%%%%%%%%%%%%%%%%%%%%%%%%%%%%%%%%%%%
\subsubsection{Transformation of $\eta_\delta$ under $\SL2Z$}
\label{sec:eta_delta-transformation}
%%%%%%%%%%%%%%%%%%%%%%%%%%%%%%%%%%%%%%%%%%%%%%%%%%%%%%%%%%%%%%%%%%%
Let $\delta \in \setN\setminus\Set{0}$.

\begin{gather*}
\eta_\delta(\gamma\tau)
=
\eta_\delta\left(\frac{a\tau+b}{c\tau+d}\right) =
\eta\left(\frac{a\delta\tau+b\delta}{c\tau+d}\right)
=
\eta\left(\begin{pmatrix}a\delta&b\delta\\c&d\end{pmatrix}
  \tau\right).
\end{gather*}

Let
%
\begin{gather}
  \defineNotation[h-delta]{h_\delta}
  :=
  \gcd(\delta a,c), \label{eq:h_delta}
\end{gather}
%
$\defineNotation[a-delta]{a_\delta}:=\frac{\delta a}{h_\delta}$,
%
$\defineNotation[c-delta]{c_\delta}:=\frac{c}{h_\delta}$,
%
and $\defineNotation[b-delta]{b_\delta}$ and
$\defineNotation[d-delta]{d_\delta}$ are chosen in such a way that
$a_\delta d_\delta - b_\delta c_\delta = 1$.
%
Because of $\gcd(a_\delta, c_\delta)=1$, such $b_\delta$ and $d_\delta$
can be found.
%
Therefore
  $\defineNotation[gamma-delta]{\gamma_\delta} := \left(
  \begin{smallmatrix}a_\delta&b_\delta\\c_\delta&d_\delta\end{smallmatrix}
\right)\in \SL2Z$.

Note that $h_\delta=\gcd(\delta, c)$, because $\gcd(a,c)=1$ and,
furthermore,
\begin{align}
\begin{pmatrix}a\delta&b\delta\\c&d\end{pmatrix}
&=\begin{pmatrix}a_\delta&b_\delta\\c_\delta&d_\delta\end{pmatrix}
  \begin{pmatrix}h_\delta&\delta b d_\delta-d b_\delta\\
                 0      &\delta / h_\delta\end{pmatrix}.
\label{eq:naive-matrix-split}
\end{align}
Note that for any integer $s$ whe have:
\begin{align*}
\begin{pmatrix}a\delta&b\delta\\c&d\end{pmatrix}
&=\begin{pmatrix}a_\delta&b_\delta\\c_\delta&d_\delta\end{pmatrix}
  \begin{pmatrix}1& s\\0&1\end{pmatrix}
  \begin{pmatrix}1&-s\\0&1\end{pmatrix}
  \begin{pmatrix}
    h_\delta&\delta b d_\delta-d b_\delta\\
    0      &\delta / h_\delta\end{pmatrix}
  \\
&=
  \begin{pmatrix}
    a_\delta&b_\delta + a_\delta s\\
    c_\delta&d_\delta + c_\delta s
  \end{pmatrix}
  \begin{pmatrix}h_\delta&\delta b d_\delta-d b_\delta-s \delta / h_\delta\\
                 0      &\delta / h_\delta\end{pmatrix}.
\end{align*}
Therefore, we can choose $b_\delta$ and $d_\delta$ in such a way that
$0 \le \delta b d_\delta-d b_\delta < \delta/h_\delta$.

If $c=0$, then $a=d=1$, $h_\delta=\delta$. We can choose $d_\delta=1$
and $b_\delta= \delta b  d_\delta$ and get
\begin{align}
\begin{pmatrix}\delta&b\delta\\0&1\end{pmatrix}
&=\begin{pmatrix}1&b \delta\\0&1\end{pmatrix}
  \begin{pmatrix}\delta&0\\
                 0      &1\end{pmatrix}.
\end{align}

If we set
\begin{gather}
  \defineNotation[tau-delta]{\tau_\delta}
  :=
  \begin{pmatrix}
    h_\delta & \delta b d_\delta-d b_\delta\\
    0       & \delta / h_\delta
  \end{pmatrix} \tau
  =
  \frac{h_\delta\tau+\delta b d_\delta-d b_\delta}{\delta/h_\delta},
  \label{eq:tau_delta}
\end{gather}
then
\begin{gather*}
  \eta_\delta(\gamma\tau)
  =
  \eta(\gamma_\delta \tau_\delta)
  =
  (c_\delta\tau_\delta+d_\delta)^{1/2}
  \,
  \upsilon(\gamma_\delta)
  \,
  \eta(\tau_\delta).
\end{gather*}


From Lemma~\ref{thm:c*tau+d} we get
\begin{align*}
  c_\delta\tau_\delta+d_\delta
  &=
  \frac{h_\delta}{\delta}(c \tau + d).
\end{align*}

Thus, we have
\begin{gather}
\eta_\delta(\gamma\tau)
=
\left(\frac{h_\delta}{\delta}(c \tau+d)\right)^{\!1/2}
\unityPower{\frac{\kappa_\delta}{24}}
\,
\eta(\tau_\delta).
\label{eq:eta_delta(gamma*tau)}
\end{gather}
where
$\defineNotation[kappa-delta]{\kappa_\delta} \in \Set{0,\ldots,23}$ is
defined by
\begin{gather}
  \label{eq:kappa_delta}
  \unityPower{\frac{\kappa_\delta}{24}} := \upsilon(\gamma_\delta).
\end{gather}
Note that for
$\gamma=\left(\begin{smallmatrix}1&0\\0&1\end{smallmatrix}\right)$ we
have $\kappa_\delta=0$ for any $\delta$.


For the following definition see \cite[Lemma~2.37]{Radu:PhD:2010} and
\cite[Definition~2.9]{Radu:PhD:2010}.

%%%%%%%%%%%%%%%%%%%%%%%%%%%%%%%%%%%%%%%%%%%%%%%%%%%%%%%%%%%%%%%%%%%
\begin{Definition}\label{def:width0}
  Let
  $\gamma=\left(\begin{smallmatrix}a&b\\c&d\end{smallmatrix}\right)
  \in \SL2Z$ and $N$ be a positive integer. Then
  \begin{gather}
    \defineNotation[w-gamma]{w_\gamma} = \frac{N}{\gcd(c^2, N)}
    \label{eq:width0}
  \end{gather}
  is called the \emph{width of $\gamma$ with respect to $\Gamma_0(N)$}.
\end{Definition}
%%%%%%%%%%%%%%%%%%%%%%%%%%%%%%%%%%%%%%%%%%%%%%%%%%%%%%%%%%%%%%%%%%%

Note that for
$\gamma=\left(\begin{smallmatrix}1&0\\0&1\end{smallmatrix}\right)$ we
have $w_\gamma=1$.


For
$\gamma =
\bigl(
\begin{smallmatrix}
  a & b\\
  c & d
\end{smallmatrix}
\bigr) \in \SL2Z$ we want to expand $\eta(\tau_\delta)$ in
$\defineNotation[x]{x}:=q^{1/w_\gamma}=\unityPower{\frac{\tau}{w_\gamma}}$
with coefficients from $\setQ(\xi)$ where
$\defineNotation[xi]{\xi} := \unityPower{\frac{c}{24N}}$.

Since we are only interested in expansions at the cusps, we can assume
$0<c\in\Delta$.

In the following let $\delta \in \Delta$. Because of
$x=q^{1/w_\gamma}$, \eqref{eq:tau_delta} and \eqref{eq:width0}, we
have
\begin{gather}
  \defineNotation[q-delta]{q_\delta} = \unityPower{\tau_\delta}
  = q^{u_\delta} \, \unityPower{v_\delta}
  = x^{e_\delta} \, \unityPower{v_\delta}
\label{eq:q_delta}
\end{gather}
for
\begin{gather}
  \defineNotation[u-delta]{u_\delta} :=
  \frac{h_\delta^2}{\delta},
  \qquad
  \defineNotation[e-delta]{e_\delta} :=
  w_\gamma u_\delta,
  \qquad
  \defineNotation[v-delta]{v_\delta} :=
  \frac{\delta b d_\delta-d b_\delta}{\delta/h_\delta}.
  \label{eq:uv_delta}
\end{gather}
Note that for
$\gamma=\left(\begin{smallmatrix}1&0\\0&1\end{smallmatrix}\right)$,
i.e., if $c=0$, then $a=d=1$ and
$w_\gamma=1$,
$h_\delta=\delta$, $u_\delta=e_\delta=\delta$, $v_\delta=0$.

%
Note that $\unityPower{\frac{v_\delta}{24}} = \xi^k$ for
$k = (\delta b d_\delta - d b_\delta) \frac{N}{\lcm(\delta,c)}\in \setZ$.

%%%%%%%%%%%%%%%%%%%%%%%%%%%%%%%%%%%%%%%%%%%%%%%%%%%%%%%%%%%%%%%%%%%
\begin{Lemma}
  If $\gamma =
\bigl(
\begin{smallmatrix}
  a & b\\
  c & d
\end{smallmatrix}
\bigr) \in \SL2Z$, $N > 0$, $\delta \in \Delta$, then
$e_\delta \in \setZ$.
\end{Lemma}
%%%%%%%%%%%%%%%%%%%%%%%%%%%%%%%%%%%%%%%%%%%%%%%%%%%%%%%%%%%%%%%%%%%
\begin{proof}
If $c=0$, then $e_\delta=\delta\in\setZ$. Without loss of generality,
we can assume $c>0$.
If $p$ is a prime that divides $N$, \ie, $N=N'p^\alpha$ for some
$\alpha>0$, and $\delta = \delta' p^m$, $c=c' p^k$ with
$\gcd(p,N')=\gcd(p,\delta')=\gcd(p,c')=1$, then
\begin{align*}
  e_\delta
  &=
  \frac{N}{\gcd(c^2,N)} \frac{\gcd(\delta,c)^2}{\delta}\\
  &=
  \frac{p^\alpha N' \gcd(p^m \delta', p^k c')^2}{\gcd(p^{2k}
    c'^2,p^\alpha N') p^m \delta'}\\
  &=
  p^{\alpha + 2 \min(m,k) - m - \min(2k, \alpha)}
  \frac{N' \gcd(\delta', c')^2}{\gcd(c'^2, N') \delta'}
\end{align*}
If we can show that $e:=\alpha + 2 \min(m,k) - m - \min(2k,
\alpha)\ge0$ then $e_\delta\in\setN$ follows by repeating the above
process for every prime divisor of $N$.

There are several cases to consider:
\begin{itemize}
\item $0\le m\le k \le \alpha$, $2k \le \alpha$. Then
  $e=\alpha+2m-m-2k=(\alpha-2k)+m\ge0$.
\item $0\le m\le k \le \alpha < 2k$. Then
  $e=\alpha+2m-m-\alpha=m\ge0$.

\item $0\le k\le m \le \alpha$, $2k \le \alpha$. Then
  $e=\alpha+2k-m-2k=\alpha-m\ge0$.
\item $0\le k\le m \le \alpha < 2k$. Then
  $e=\alpha+2k-m-\alpha=2k-m\ge0$.
\end{itemize}
\end{proof}
%%%%%%%%%%%%%%%%%%%%%%%%%%%%%%%%%%%%%%%%%%%%%%%%%%%%%%%%%%%%%%%%%%%

We can expand $\eta(\tau_\delta)$ in terms of $x$ as follows.
\begin{gather*}
  \eta(\tau_\delta)
  =
    \unityPower{{\frac{\tau_\delta}{24}}}
    \prod_{n=1}^{\infty}(1-q_\delta^n)
  =
  \unityPower{\frac{v_\delta}{24}} x^{e_\delta/24}
  \cdot
  \prod_{n=1}^{\infty}(1-q_\delta^n).
\end{gather*}

Then \eqref{eq:eta_delta(gamma*tau)} turns into
\begin{align}
\eta_\delta(\gamma\tau)
&=
(c \tau+d)^{1/2}
\,
\left(\frac{h_\delta}{\delta}\right)^{\!\frac{1}{2}}
\,
  \unityPower{\frac{v_\delta + \kappa_\delta}{24}}
  x^{e_\delta/24} \cdot \prod_{n=1}^{\infty}(1-q_\delta^n)
\label{eq:eta_delta(gamma*tau)-expansion}
\end{align}
where the values are given through \eqref{eq:h_delta},
\eqref{eq:kappa_delta}, \eqref{eq:width0}, \eqref{eq:q_delta},
\eqref{eq:uv_delta}, and $x=\unityPower{\frac{\tau}{w_\gamma}}$.



The parts of \eqref{eq:eta_delta(gamma*tau)} are implemented via
\textcolor{blue}{\code{SymbolicEtaGamma}}
(Section~\ref{sec:SymbolicEtaGamma}).
\\
If \code{e = eta(nn, delta, gamma)}, then we have the following
correspondence.
\begin{align*}
\text{\code{rationalPrefactor(e) : Q}} &= \frac{h_\delta}{\delta},\\
\text{\code{edelta(e) : Z}}            &= e_\delta,\\
\text{\code{udelta(e) : Q}}            &= e_\delta,\\
\text{\code{vdelta(e) : Q}}            &= v_\delta,\\
\text{\code{upsilonExponent(e) : Z}}   &= \kappa_\delta.
\end{align*}

%%%%%%%%%%%%%%%%%%%%%%%%%%%%%%%%%%%%%%%%%%%%%%%%%%%%%%%%%%%%%%%%%%%
\begin{Definition}
  For any $c, \delta \in\Delta$, $r\in R(N)$ let us define
  \begin{align}
    \defineNotation[a-N]{a_N}(c, \delta)
    &:= \frac{N}{\gcd(N, c^2)} \frac{\gcd(c,\delta)^2}{\delta},
      \notag\\
    \defineNotation[ord-c-N]{\ord_c^N}(r)
    &:= \frac{1}{24}\divisorsum{N} a_N(c, \delta) \, r_\delta.
    \label{eq:order-r}
  \end{align}
\end{Definition}

With $\defineNotation[ord-gamma-N]{\ord_\gamma^N}$ as defined in
\cite{Radu:RamanujanKolberg:2015}, Theorem~23 of
\cite{Radu:RamanujanKolberg:2015} turns into
%%%%%%%%%%%%%%%%%%%%%%%%%%%%%%%%%%%%%%%%%%%%%%%%%%%%%%%%%%%%%%%%%%%
\begin{Theorem}\label{thm:order}
  Let $\gamma =
  \bigl(
  \begin{smallmatrix}
    a & b\\
    c & d
  \end{smallmatrix}
  \bigr)
  \in \SL2Z$ with $c\in\Delta$.
  If $r \in R^*(N)$, then
  $\ord_\gamma^N(g_r) = \ord_c^N(r)$.
\end{Theorem}
%%%%%%%%%%%%%%%%%%%%%%%%%%%%%%%%%%%%%%%%%%%%%%%%%%%%%%%%%%%%%%%%%%%
For a proof we refer to
\cite[Proposition~3.2.8]{Ligozat:CourbesModulaires:1975}.

In the following let us fix $r \in R(N)$ and
$\gamma =
  \bigl(
  \begin{smallmatrix}
    a & b\\
    c & d
  \end{smallmatrix}
  \bigr)
  \in \SL2Z$ with $c \in \Delta$.

From \eqref{eq:eta_delta(gamma*tau)-expansion} follows
\begin{align}
  \defineNotation[g-r-gamma-tau]{g_r(\gamma \tau)}
  &=
  \divisorprod{N}
  (c \tau+d)^{r_\delta/2}
  \,
  \left(\frac{h_\delta}{\delta}\right)^{\!r_\delta/2}
  \,
  \unityPower{\frac{r_\delta(v_\delta + \kappa_\delta)}{24}}
  \cdot
  q^{r_\delta u_\delta/24}
  \cdot
    \prod_{n=1}^{\infty}(1-q_\delta^n)^{r_\delta}
  \notag
  \\
  %
  &=
  (c \tau+d)^{\divisorsum{N} \frac{r_\delta}{2}}
  \cdot
  \divisorprod{N}
  \left(\frac{h_\delta}{\delta}\right)^{\!r_\delta/2}
  \cdot
  \unityPower{\divisorsum{N} \frac{r_\delta(v_\delta + \kappa_\delta)}{24}}
  \cdot
  q^{\divisorsum{N} \frac{r_\delta u_\delta}{24}}
  \cdot
  \divisorprod{N} (q_\delta; q_\delta)^{r_\delta}.
  \label{eq:g_r(gamma*tau)}
\end{align}

Note that for
$\gamma=\left(\begin{smallmatrix}1&0\\0&1\end{smallmatrix}\right)$ the
first three products are equal to 1.


The parts of \eqref{eq:g_r(gamma*tau)} are implemented via
\textcolor{blue}{\code{SymbolicEtaQuotientGamma}}
(Section~\ref{sec:SymbolicEtaQuotientGamma}).
\\
If \code{e = eta(nn, r, gamma)}, then we have the following
correspondence.
\begin{align*}
  \text{\code{rationalPrefactor(e) : Q}}
  &=
    \divisorprod{N} \left(\frac{h_\delta}{\delta}\right)^{\!r_\delta},
  \\
  \text{\code{unityExponent(e) : Q}}
  &=
    \divisorsum{N} \frac{r_\delta(v_\delta + \kappa_\delta)}{24},
  \\
  \text{\code{xExponent(e) : Q}}
  &=
    \divisorsum{N} \frac{r_\delta e_\delta}{24},
  \\
  \text{\code{qExponent(e) : Q}}
  &=
    \divisorsum{N} \frac{r_\delta u_\delta}{24},
\end{align*}

If $r\in R^*(N)$, then $g_r$ is a modular function on $\Gamma_0(N)$.
Because of \eqref{eq:sum=0}, $q^{u_\delta}=x^{e_\delta}$,
$e_\delta=a_N(c, \delta)$, and \eqref{eq:order-r}, we can write
\begin{align}
  g_r(\gamma \tau)
  &=
  \divisorprod{N}
  \left(\frac{h_\delta}{\delta}\right)^{\!r_\delta/2}
  \cdot
  \unityPower{\divisorsum{N} \frac{r_\delta(v_\delta + \kappa_\delta)}{24}}
  \cdot
  x^{\ord^N_c(r)}
  \cdot
  \divisorprod{N} (q_\delta; q_\delta)^{r_\delta}.\\
  \label{eq:modular-g_r(gamma*tau)}
\end{align}
Thus, we can expand $g_r(\gamma\tau)$ as a Laurent series in
$x:=q^{1/w_\gamma}$ with coefficients from $\setQ(\xi)$.


%%%%%%%%%%%%%%%%%%%%%%%%%%%%%%%%%%%%%%%%%%%%%%%%%%%%%%%%%%%%%%%%%%%
\subsubsection{Root of unity reduction}
%%%%%%%%%%%%%%%%%%%%%%%%%%%%%%%%%%%%%%%%%%%%%%%%%%%%%%%%%%%%%%%%%%%

Note that any modular function (in particular $g_r$) can be expressed
as a rational function $f$ in $j$ and $j_N$ where
$j_N(\tau):=j(N\tau)$ and $j$ is Klein's $j$-invariant, \ie,
$g_r(\tau)=f(j, j_N)(\tau) := f(j(\tau), j(N\tau))$.

Let $\frac{a}{c}$ be a cusp of $\Gamma_0(N)$.
%
We can choose $b$ and $d$ such that
%
$\gamma := \bigl(
\begin{smallmatrix}
  a & b\\
  c & d
\end{smallmatrix}
\bigr) \in \SL2Z$ and $\gamma\infty=\frac{a}{c}$.
%
Klein's $j$-function is invariant under any modular transformation.
%
Let us consider the transformation of $j_N$ by $\gamma$.
%
According to \eqref{eq:tau_delta} for $\delta=N$, $j_N(\gamma\tau)$
can be expanded into a Laurent series in
$x=q^{1/w_\gamma}=\unityPower{\tau/w_\gamma}$ with
coefficients from $\setQ(\xi)$ where $\xi$ is a $\frac{N}{c}$-th root
of unity.
%
Since there is some freedom to choose such $b$ and $d$, we show in the
following, how we can obtain an expansion of $j_N(\gamma\tau)$ and
therefore (via $f$) an expansion $g_r(\gamma\tau)$ into a Laurent
series in $x$ with coefficients in $\setQ(\xi)$ where $\xi$ is a
$v$-th root of unity and $1 \leq v \le N/c$ and $v<N/c$ if $c<N$.

Let
$\gamma' = \bigl(
\begin{smallmatrix}
  a & b'\\
  c & d'
\end{smallmatrix}
\bigr) \in \SL2Z$, then also
\begin{align*}
  \gamma
  &:=
    \begin{pmatrix}
      a & b' + as\\
      c & d' + cs
    \end{pmatrix}
  =
    \begin{pmatrix}
      a & b'\\
      c & d'
    \end{pmatrix}
    \begin{pmatrix}
      1 & s\\
      0 & 1
    \end{pmatrix}
    \in \SL2Z
\end{align*}
for any $s\in\setZ$.

For the transformation of $j_N$, we split the matrix as in
\eqref{eq:naive-matrix-split} with $\delta=N$.
\begin{align*}
  \begin{pmatrix}
    N a & N(b' + as)\\
    c & d' + cs
  \end{pmatrix}
      &=
  \begin{pmatrix}
    N a/c & -1\\
    1     &  0
  \end{pmatrix}
  \begin{pmatrix}
    c & cs+d'\\
    0 & N/c
  \end{pmatrix}
\end{align*}
into an element of $SL_2(Z)$ and a triangular matrix.
%
Then we determine natural numbers $u$ and $v$ such that $u v = N/c$
and $\gcd(c, u)=1$. Thus, we can find $s \in \setZ$ such that
$s \equiv -d'c^{-1} \pmod{u}$, \ie, $cs + d' = tu$ for some integer
$t$ with $0\le t<c$.
%
Thus,
\begin{gather*}
  \begin{pmatrix}
    c & cs+d'\\
    0 & N/c
  \end{pmatrix} \tau
  =
  \begin{pmatrix}
    c & u t\\
    0 & u v
  \end{pmatrix} \tau
  =
  \frac{c^2}{N} \tau + \frac{t}{v}
  =
 \frac{w_\gamma c^2}{N} \frac{\tau}{w_\gamma} + \frac{t}{v}
  =
 \frac{c}{\gcd(c,N/c)} \frac{\tau}{w_\gamma} + \frac{t}{v}
  .
\end{gather*}
Note that $u=N/c$ and $v=1$, if $N$ is squarefree.

With the particular $s$ from above, we define $b:=b'+as$, $d:=d'+cs$
and take
%
$\gamma = \bigl(
\begin{smallmatrix}
  a & b\\
  c & d
\end{smallmatrix}
\bigr)$ as the transformation matrix that belongs to the cusp
$\frac{a}{c}$.


%%%%%%%%%%%%%%%%%%%%%%%%%%%%%%%%%%%%%%%%%%%%%%%%%%%%%%%%%%%%%%%%%%%
\subsection{The function $g_{r,m,t}$ and its transformations under
  $\SL2Z$}
  \label{sec:function-g_r-m-t}
%%%%%%%%%%%%%%%%%%%%%%%%%%%%%%%%%%%%%%%%%%%%%%%%%%%%%%%%%%%%%%%%%%%

In the following let
$\defineNotation[m]{m},
\defineNotation[M]{M},
N \in \setN \setminus \Set{0}$, and
$r \in R(M)$
be subject to the following conditions
\begin{gather}
  \text{for every prime $p$ with $\divides{p}{m}$ follows $\divides{p}{N}$,}
  \label{eq:p|m=>p|N}\\
  \text{and, if $\divides{\delta}{M}$ and $r_\delta\ne0$, then
    $\divides{\delta}{mN}$.}
  \label{eq:delta|M=>delta|mN}
\end{gather}

For the above conditions see (13) and (14) in
\cite{Radu:AlgorithmicApproachRamanujanCongruences:2009} or
Section~4.2 in \cite{Radu:PhD:2010}.

Furthermore, we fix $t\in\Set{0,\ldots,m-1}$.

%%%%%%%%%%%%%%%%%%%%%%%%%%%%%%%%%%%%%%%%%%%%%%%%%%%%%%%%%%%%%%%%%%%

Let
\begin{gather*}
  f(\tau)
  =
  \sum_{n=0}^\infty a(n) q^n
  =
  \divisorprod{M}\prod_{n=1}^\infty(1-q^{\delta n})^{r_\delta}
  =
  \divisorprod{M}\eulerFunction{\delta}^{r_\delta}
  =
  \unityPower{-\frac{\tau \, \sumdelta{r}}{24}} g_r(\tau)
\end{gather*}
be the generating function for the sequence $(a(n))_{n\in\setN}$
where we have abbreviated
\begin{gather}
  \sumdelta{r} := \divisorsum{M} \delta r_\delta.
  \label{eq:sumdelta}
\end{gather}

Let $\defineNotation[U-m]{U_m}$ be an operation on functions
$\setH\to\setC$ so that
\begin{gather}
  (U_m\phi)(\tau) := \frac{1}{m}\sum_{\lambda=0}^{m-1}
  \phi\left(\frac{\tau+\lambda}{m}\right).
  \label{eq:U_m}
\end{gather}
Let
\begin{gather}
  f_t(\tau)
  :=\unityPower{-t\tau}f(\tau)
  = \unityPower{-\frac{\tau (24 t + \sumdelta{r})}{24}} g_r(\tau),
  \label{eq:f_t}
\end{gather}
then
\begin{align}
  (U_mf_t)(\tau)
  &=
  \frac{1}{m} \sum_{\lambda=0}^{m-1}f_t\left(\frac{\tau+\lambda}{m}\right)
  \notag
  \\
  &=
  \frac{1}{m} \sum_{\lambda=0}^{m-1}
  \unityPower{-\frac{t\tau+t\lambda}{m}}
  \sum_{k=0}^\infty a(k) \unityPower{\frac{k\tau+k\lambda}{m}}
  \notag
  \\
  &=
  \frac{1}{m}
  \sum_{k=0}^\infty
  a(k)
  \sum_{\lambda=0}^{m-1}
  \unityPower{\frac{(k-t)(\tau+\lambda)}{m}}
  \notag
  \\
  &=
  \frac{1}{m}
  \sum_{k=0}^\infty
  a(k)\,
  \unityPower{\frac{(k-t)\tau}{m}}
  \sum_{\lambda=0}^{m-1}
  \unityPower{\frac{(k-t)\lambda}{m}}
  \notag
  \\
  &=
    \sum_{n=0}^\infty
    \sum_{l=0}^{m-1} a(mn+l)\,
    \unityPower{\frac{(mn+l-t)\tau}{m}}
    \frac{1}{m}
    \sum_{\lambda=0}^{m-1}\unityPower{\frac{(mn+l-t)\lambda}{m}}
    \notag
  \\
  &=
    \sum_{n=0}^\infty a(mn+t) q^n
  \label{eq:U_m-f_t}
\end{align}
is the generating function for the series
$(a(mn+t))_{n\in\setN}$.


Note that
$\sum_{\lambda=0}^{m-1}\unityPower{\frac{(mn+l-t)\lambda}{m}}$ is
equal to $m$ for $l=t$ and vanishes in all other cases.

We can also evaluate $(U_mf_t)(\tau)$ in another way.
%
\begin{align*}
  (U_mf_t)(\tau)
  &=
  \frac{1}{m} \sum_{\lambda=0}^{m-1}f_t\left(\frac{\tau+\lambda}{m}\right)\\
  &=
    \frac{1}{m} \sum_{\lambda=0}^{m-1}
    \unityPower{-\frac{(\tau+\lambda) (24 t + \sumdelta{r})}{24 m}}
    g_r\!\left(\frac{\tau+\lambda}{m}\right)
  \\
% &=
% \frac{1}{m} \sum_{\lambda=0}^{m-1}
% \unityPower{-\frac{t\tau+t\lambda}{m} - \frac{\tau+\lambda}{24m} \sumdelta{r}}
% g_r\left(\frac{\tau+\lambda}{m}\right)\\
  &=
  \frac{1}{m}
  \unityPower{-\frac{\tau (24t+\sumdelta{r})}{24m}}
  \sum_{\lambda=0}^{m-1}
  \unityPower{-\frac{\lambda (24t+\sumdelta{r})}{24m}}
  \divisorprod{M}
    \eta\left(\frac{\delta(\tau+\lambda)}{m}\right)^{r_\delta}
\end{align*}

As abbreviation, we define the function
\begin{align}
  \defineNotation[g-r-m-t-lambda-tau]{g_{r,m,t,\lambda}(\tau)}
  &:=
  \unityPower{-\frac{\lambda}{24m} (24t+\sumdelta{r})}
  \divisorprod{M}
  \eta\left(\frac{\delta(\tau+\lambda)}{m}\right)^{r_\delta}
  =
  \unityPower{-\frac{\lambda}{24m} (24t+\sumdelta{r})}
  g_r\!\left(\frac{\tau+\lambda}{m}\right)
  \label{eq:g_r-m-t-lambda(tau)}
\end{align}

In the following we consider the function
\begin{align}
  \defineNotation[g-r-m-t-tau]{g_{r,m,t}(\tau)}
  &:= \unityPower{\frac{\tau (24t+\sumdelta{r})}{24m} } (U_mf_t)(\tau)
  =
  \frac{1}{m} \sum_{\lambda=0}^{m-1} g_{r,m,t,\lambda}(\tau)
  \label{eq:g_r-m-t(tau)}
\end{align}
and its transformations under $\SL2Z$.

Note that $g_r = g_{r, 1, 0}$.

We aim at the expansion of $g_{r,m,t}(\tau)$ at all cusps
$\gamma = \left(
  \begin{smallmatrix}a&b\\c&d\end{smallmatrix} \right)\in \SL2Z$
of $\Gamma_0(N)$ into Puiseux series in $x=q^{1/w_\gamma}$ or rather
into Laurent series in
$\defineNotation[z]{z} := q^{\frac{1}{24 w_\gamma}}$. Temporarily (for
the expansion of $\eta_{\delta,m,\lambda}$) we will expand into
Laurent series in
$\defineNotation[z-m]{z_m} := q^{\frac{1}{24 m w_\gamma}}$.

%%%%%%%%%%%%%%%%%%%%%%%%%%%%%%%%%%%%%%%%%%%%%%%%%%%%%%%%%%%%%%%%%%%
\subsubsection{Transformation of $\eta_{\delta,m,\lambda}$ under
  $\SL2Z$}
  \label{sec:transformation-eta_delta-m-lambda}
%%%%%%%%%%%%%%%%%%%%%%%%%%%%%%%%%%%%%%%%%%%%%%%%%%%%%%%%%%%%%%%%%%%

Let $\defineNotation[lambda]{\lambda} \in \Set{0,\ldots,m-1}$, and
$\divides{\delta}{M}$.
%
We define
\begin{gather}
  \defineNotation[eta-delta-m-lambda-tau]{\eta_{\delta,m,\lambda}}(\tau)
  := \eta \left(
    \begin{pmatrix}\delta& \delta \lambda\\0&m\end{pmatrix} \tau
  \right).
\label{eq:eta_delta-m-lambda}
\end{gather}

Let
$\gamma=\left(\begin{smallmatrix}a&b\\c&d\end{smallmatrix}\right) \in
\SL2Z$.
%
Then
\begin{align}
  \eta_{\delta,m,\lambda}(\gamma \tau)
  &:= \eta
\left(
  \begin{pmatrix}\delta& \delta \lambda\\0&m\end{pmatrix}
  \gamma
  \tau
\right)
=
\eta(W \tau)
\end{align}
where
\begin{align}
  \defineNotation[W]{W}
  &:=
  \begin{pmatrix}
    \delta (a + c \lambda) & \delta (b + d \lambda)\\
    c m                   & d m
  \end{pmatrix}.
  \label{eq:W_delta-m-lambda}
\end{align}

Setting $z = \delta \lambda$ in Lemma~\ref{thm:c*tau+d}, we get the
splitting of $W = W' W''$ as
%
\begin{gather}
  \defineNotation[W']{W'}
  =
  \begin{pmatrix}
    a' & b'\\
    c' & d'
  \end{pmatrix}
  = \begin{pmatrix}
      \delta(a+c \lambda)/h_{\delta,m,\lambda} & b'\\
      cm/h_{\delta,m,\lambda}                  & d'
    \end{pmatrix},
  \qquad
  \defineNotation[W'']{W''}
  =
  \begin{pmatrix}
    h_{\delta,m,\lambda} & \delta (b + d \lambda) d' - d m b'\\
    0               & \delta m / h_{\delta,m,\lambda}
  \end{pmatrix}
  \label{eq:split-W_delta-m-lambda}
\end{gather}
where $b'$ and $d'$ have been chosen in such a way that
$0\le  \delta (b + d \lambda) d' < \delta m / h_{\delta,m,\lambda}$.
and
\begin{gather}
  \defineNotation[h-delta-m-lambda]{h_{\delta,m,\lambda}} :=
  \gcd(\delta (a + c \lambda), cm).
\end{gather}

Let us define
\begin{gather}
  \defineNotation[u-delta-m-lambda]{u_{\delta,m,\lambda}}
  :=
  \frac{h_{\delta,m,\lambda}^2}{\delta m},
  \qquad
  \defineNotation[v-delta-m-lambda]{v_{\delta,m,\lambda}}
  :=
  \frac{h_{\delta,m,\lambda}(\delta (b + d \lambda) d' - d m b')}{\delta m}.
\end{gather}

And further,
\begin{align}
  \defineNotation[gamma-delta-m-lambda]{\gamma_{\delta,m,\lambda}}
  &:=
    W'\in\SL2Z,
    \label{eq:gamma_delta-m-lambda}
  \\
  \defineNotation[tau-delta-m-lambda]{\tau_{\delta,m,\lambda}}
  &:=
    W''\tau
    = \frac{h_{\delta,m,\lambda}^2 \tau}{\delta m} +
    \frac{h_{\delta,m,\lambda}(\delta (b + d \lambda) d' - d m
    b')}{\delta m}
  \notag\\
    &= u_{\delta,m,\lambda} \tau + v_{\delta,m,\lambda}.
  \label{eq:tau_delta-m-lambda}
\end{align}
Then
$\defineNotation[q-delta-m-lambda]{q_{\delta,m,\lambda}} :=
\unityPower{\tau_{\delta,m,\lambda}} =
q^{u_{\delta,m,\lambda}}\,\unityPower{v_{\delta,m,\lambda}}$.


Lemma~\ref{thm:c*tau+d} tells us
$c'\tau_{\delta,m,\lambda}+d' = \frac{h_{\delta,m,\lambda}}{\delta} (c \tau + d)$.
%
Therefore,
\begin{align}
  \defineNotation[eta_delta-m-lambda-gamma-tau]{\eta_{\delta,m,\lambda}(\gamma \tau)}
  &= \eta(W \tau)
    = (c'\tau'+d')^{1/2} \, \upsilon(\gamma_{\delta,m,\lambda})
    \,\eta(\tau_{\delta,m,\lambda})\notag\\
  &=
    (c\tau+d)^{1/2}
    \left(\frac{h_{\delta,m,\lambda}}{\delta}\right)^{1/2}
    \unityPower{\frac{\kappa_{\gamma_{\delta,m,\lambda}}}{24}} \,
    q_{\delta,m,\lambda}^{1/24}
    \prod_{n=1}^\infty (1-q_{\delta,m,\lambda}^n)\notag\\
  &=
    (c\tau+d)^{1/2}
    \left(\frac{h_{\delta,m,\lambda}}{\delta}\right)^{1/2}
    \unityPower{\frac{ v_{\delta,m\lambda} + \kappa_{\gamma_{\delta,m,\lambda}}}{24}}
    q^{u_{\delta,m,\lambda}/24}
    \prod_{n=1}^\infty (1-q_{\delta,m,\lambda}^n)
    \label{eq:eta_delta-m-lambda(gamma*tau)}
\end{align}
where
$\defineNotation[kappa-gamma-delta-m-lambda]{\kappa_{\gamma_{\delta,m,\lambda}}}
\in \Set{0,\ldots,23}$ is defined by
$\unityPower{\frac{\kappa_{\gamma_{\delta,m,\lambda}}}{24}} =
\upsilon(\gamma_{\delta,m,\lambda})$.

The parts of \eqref{eq:eta_delta-m-lambda(gamma*tau)} are implemented
via \textcolor{blue}{\code{SymbolicEtaGamma}}
(Section~\ref{sec:SymbolicEtaGamma}).
\\
If \code{e = eta(mm, delta, m, lambda, gamma)}, then we have the
following correspondence.
\begin{align*}
  \text{\code{rationalPrefactor(e) : Q}}
  &=
    \frac{h_{\delta,m,\lambda}}{\delta},
  \\
  \text{\code{udelta(e) : Q}}
  &=
    u_{\delta,m,\lambda},
  \\
  \text{\code{vdelta(e) : Q}}
  &=
    v_{\delta,m,\lambda},
  \\
  \text{\code{upsilonExponent(e) : Z}}
  &=
    \kappa_{\delta,m,\lambda}.
\end{align*}



We expand $\eta_{\delta,m,\lambda}(\gamma \tau)$ into a Laurent series
in $z_m$ with coefficients from $\setQ(\xi_m)$ where
$\xi_m := \unityPower{\frac{1}{24 m N}}$. Note that
$\unityPower{\frac{v_{\delta,m,\lambda}}{24}}=\xi_m^k$ for
$k=\frac{N}{\delta} h_{\delta,m,\lambda}(\delta (b + d \lambda) d' - d
m b') \in \setZ$.
\begin{Hemmecke}
  $\xi_m$ is not yet what we want.
\end{Hemmecke}

%%%%%%%%%%%%%%%%%%%%%%%%%%%%%%%%%%%%%%%%%%%%%%%%%%%%%%%%%%%%%%%%%%%
\subsubsection{Transformation of $g_{r,m,t,\lambda}$ under $\SL2Z$}
%%%%%%%%%%%%%%%%%%%%%%%%%%%%%%%%%%%%%%%%%%%%%%%%%%%%%%%%%%%%%%%%%%%

From its definition~\eqref{eq:g_r-m-t-lambda(tau)} and
\eqref{eq:eta_delta-m-lambda(gamma*tau)}, we can easily find a formula
for the transformation of $g_{r,m,t,\lambda}$ under $\SL2Z$.

\begin{align}
  \defineNotation[g-r-m-t-lambda-gamma-tau]{g_{r,m,t,\lambda}(\gamma\tau)}
  &=
    \unityPower{-\frac{\lambda}{24m}\Bigl(24t+\sumdelta{r}\Bigr)}
    \divisorprod{M}
    \eta_{\delta,m,\lambda}(\gamma\tau)^{r_\delta}\notag\\
  \begin{split}
  &=
    (c\tau+d)^{\divisorsum{M} \frac{r_\delta}{2}} \cdot
    \unityPower{-\frac{\lambda}{24m}\Bigl(24t+\sumdelta{r}\Bigr)}
    \cdot
    \unityPower{\divisorsum{M}\frac{r_\delta(v_{\delta,m,\lambda} +
    \kappa_{\gamma_{\delta,m,\lambda}})}{24}} \times\\
  & \qquad\times
    \divisorprod{M}
    \left(\frac{h_{\delta,m,\lambda}}{\delta}\right)^{\!\frac{r_\delta}{2}}
    \cdot
    \unityPower{\frac{\divisorsum{M} r_\delta u_{\delta,m,\lambda}}{24} \tau}
    \cdot
    \divisorprod{M}
    \prod_{n=1}^\infty (1-q_{\delta,m,\lambda}^n)^{r_\delta}
  \end{split}
  \label{eq:g_r-m-t-lambda(gamma*tau)}
\end{align}

The parts of \eqref{eq:g_r-m-t-lambda(gamma*tau)} are implemented via
\textcolor{blue}{\code{SymbolicEtaGamma}}
(Section~\ref{sec:SymbolicEtaGamma}).
\\
If \code{e = eta(mm, delta, m, t, lambda, gamma)}, then we have the
following correspondence.
\begin{align*}
  \text{\code{rationalPrefactor(e) : Q}}
  &=
    \divisorprod{M}
    \left(\frac{h_{\delta,m,\lambda}}{\delta}\right)^{\!r_\delta},
  \\
  \text{\code{unityExponentPrefactor(e) : Q}}
  &=
    -\frac{\lambda}{24m}\Bigl(24t+\sumdelta{r}\Bigr)
  \\
  \text{\code{unityExponentEta(e) : Q}}
  &=
    \divisorsum{M} \frac{r_\delta(v_{\delta,m,\lambda} + \kappa_{\delta,m,\lambda})}{24}
  \\
  \text{\code{unityExponent(e) : Q}}
  &=
    -\frac{\lambda}{24m}\Bigl(24t+\sumdelta{r}\Bigr)
    +
    \divisorsum{M} \frac{r_\delta(v_{\delta,m,\lambda} + \kappa_{\delta,m,\lambda})}{24}
  \\
  \text{\code{qExponent(e) : Q}}
  &=
    \divisorsum{M} \frac{r_\delta u_{\delta,m,\lambda}}{24}
\end{align*}



%%%%%%%%%%%%%%%%%%%%%%%%%%%%%%%%%%%%%%%%%%%%%%%%%%%%%%%%%%%%%%%%%%%
\subsubsection{Transformation of $g_{r,m,t}$ under $\SL2Z$}
%%%%%%%%%%%%%%%%%%%%%%%%%%%%%%%%%%%%%%%%%%%%%%%%%%%%%%%%%%%%%%%%%%%

From its definition~\eqref{eq:g_r-m-t(tau)}
and \ref{eq:g_r-m-t-lambda(gamma*tau)}, we can easily find a formula
for the transformation of $g_{r,m,t}$ under $\SL2Z$.

\begin{align}
  \defineNotation[g-r-m-t-gamma-tau]{g_{r,m,t}(\gamma\tau)}
  &:=\frac{1}{m} \sum_{\lambda=0}^{m-1} g_{r,m,t,\lambda}(\gamma\tau)\notag\\
  \begin{split}
  &=
    \frac{(c\tau+d)^{\divisorsum{M} \frac{r_\delta}{2}}}{m}
    \sum_{\lambda=0}^{m-1}
    \unityPower{-\frac{\lambda}{24m}\Bigl(24t+\sumdelta{r}\Bigr)
    + \divisorsum{M}\frac{r_\delta(v_{\delta,m\lambda} +
    \kappa_{\gamma_{\delta,m,\lambda}})}{24}} \times\\
  & \qquad\times
    \divisorprod{M}
    \left(\frac{h_{\delta,m,\lambda}}{\delta}\right)^{\!\frac{r_\delta}{2}}
    \cdot
    \unityPower{\frac{\divisorsum{M} r_\delta u_{\delta,m,\lambda}}{24}\tau}
    \cdot
    \divisorprod{M}
    \prod_{n=1}^\infty (1-q_{\delta,m,\lambda}^n)^{r_\delta}
  \end{split}
  \label{eq:g_r-m-t(gamma*tau)}
\end{align}

%%%%%%%%%%%%%%%%%%%%%%%%%%%%%%%%%%%%%%%%%%%%%%%%%%%%%%%%%%%%%%%%%%%
\subsubsection{Find ``modular'' cofactor for $g_{r,m,t}(\tau)$}
%%%%%%%%%%%%%%%%%%%%%%%%%%%%%%%%%%%%%%%%%%%%%%%%%%%%%%%%%%%%%%%%%%%

%%%%%%%%%%%%%%%%%%%%%%%%%%%%%%%%%%%%%%%%%%%%%%%%%%%%%%%%%%%%%%%%%%%
\begin{Definition}\cite[Def.~35]{Radu:RamanujanKolberg:2015}
  \label{def:condition-co-eta-quotient-gamma0}
  Let $\Delta^*$ be the set of all tuples $(N, M, r, m, t)$ such that
  $N, M, m \in \setN\setminus\Set{0}$,
  %
  $t \in \Set{0,\ldots,m-1}$,
  %
  $r \in R(M)$
  %
  fulfil conditions \eqref{eq:p|m=>p|N} and \eqref{eq:delta|M=>delta|mN}
  and for
  %
  $\kappa:=\gcd(1-m^2, 24)\ge1$,
  %
  $w' := \gcd(\kappa(24t + \sumdelta{r}),24m)$,
  %
  $w := \frac{24m}{w'}$
  %
  the following conditions hold:
  \begin{gather}
  \kappa \frac{m N^2}{M} \divisorsum{M} \frac{M}{\delta}
    r_\delta \equiv_{24} 0,\label{eq:rv24}\\
  \kappa N \divisorsum{M} r_\delta \equiv_8 0,\label{eq:sum-r}\\
  \divides{w}{N},\label{eq:w|N}\\
  \divides{2}{m} \implies (\divides{4}{\kappa N} \land \divides{8}{N e})
  \lor
  (\divides{2}{e} \land \divides{8}{N(u-1)})\label{eq:even-m}
\end{gather}
where $e, u\in\setZ$ are such that $u$ is odd and
$\divisorprod{M} \delta^{\abs{r_\delta}}=2^e u$.
\end{Definition}

%%%%%%%%%%%%%%%%%%%%%%%%%%%%%%%%%%%%%%%%%%%%%%%%%%%%%%%%%%%%%%%%%%%
\begin{Lemma}\label{thm:y^2-1}
  Let $y\in \setZ$ be such that $gcd(y, 6)=1$ then
  $y^2 \equiv_{24} 1$, \ie, $ \frac{y^2-1}{24} \in \setZ$.
\end{Lemma}
\begin{proof}
  Let $y = 6k + l$ with $l \in \Set{0,\ldots,5}$. Then
  $\gcd(y, 6)=\gcd(l,6)=1$ only holds for $l=1$ or $l=5$.
  %
  Since $36 k^2 + 12 k l + l^2 = 12k(3k+l) + l^2$, and $12k(3k+l)$ is
  divisible by 24 for $l=1$ or $l=5$, the statement of the Lemma
  follows from the fact that $1^2\equiv_{24} 5^2 \equiv_{24} 1$.
\end{proof}

%%%%%%%%%%%%%%%%%%%%%%%%%%%%%%%%%%%%%%%%%%%%%%%%%%%%%%%%%%%%%%%%%%%
\begin{Definition}\cite[Def.~40]{Radu:RamanujanKolberg:2015}
  Let $n \in \setN$, $n>1$. For $x \in \setZ$ we denote by
  $[x]_n\in \setZ_n$ the residue class of $x$ modulo $n$ and define
  \begin{gather*}
    \defineNotation[S-n]{\setS_{n}}
    :=
    \SetDef{[y^2]_n}{y \in \setZ \land \gcd(y, n)=1}
    \subseteq
    \setZ_n.
  \end{gather*}
\end{Definition}

Let $x \in \setZ$ be such that $[x]_{24_m} \in \setS_{24m}$. Then
there exists $y \in \setZ$ with $\gcd(y, 24m)=1$ and
$[x]_{24m}=[y^2]_{24m}$, \ie, $x = y^2 + k (24m)$ for some $k\in
\setZ$. From Lemma~\ref{thm:y^2-1} follows that
$\frac{x-1}{24}=\frac{y^2-1}{24} + km \in\setZ$.


%%%%%%%%%%%%%%%%%%%%%%%%%%%%%%%%%%%%%%%%%%%%%%%%%%%%%%%%%%%%%%%%%%%
\begin{Definition}\cite[Def.~41]{Radu:RamanujanKolberg:2015}
  Let $r \in R(M)$.
  %
  The mapping
  \begin{align*}
    \defineNotation[odotbar-r]{\bar{\odot}_r}
    &: \setS_{24m} \times \Set{0,\ldots,m-1} \to \Set{0,\ldots,m-1},
    \qquad
    ([x]_{24m}, t) \mapsto [x]_{24m}  \mathbin{\bar{\odot}_r} t
  \end{align*}
  is uniquely given by
  \begin{gather}
    [x]_{24m} \mathbin{\bar{\odot}_r} t
    :=
    \left( t x + \frac{x-1}{24} \sumdelta{r} \right) \bmod{m}
  \end{gather}
\end{Definition}

%%%%%%%%%%%%%%%%%%%%%%%%%%%%%%%%%%%%%%%%%%%%%%%%%%%%%%%%%%%%%%%%%%%
\begin{Definition}
  (Definition 42 and Lemma 43 of \cite{Radu:RamanujanKolberg:2015})
  For $t\in\Set{0,\ldots,m-1}$, $r \in R(M)$ we define
  %
  \begin{align}
    \defineNotation[O-r-m-t]{\modularOrbit{r,m,t}}
    &:=
      \setS_{24m}  \mathbin{\bar{\odot}_r} t
      =
      \SetDef{x \mathbin{\bar{\odot}_r} t}{x \in \setS_{24m}},
    \\
    %
    \defineNotation[chi-r-m-t]{\chi_{r,m,t}}
    &:=
      \prod_{t' \in \modularOrbit{r,m,t}}
        \unityPower{\frac{(1-m^2)(24 t'+ \sumdelta{r})}{24 m}}.
  \end{align}
\end{Definition}

%%%%%%%%%%%%%%%%%%%%%%%%%%%%%%%%%%%%%%%%%%%%%%%%%%%%%%%%%%%%%%%%%%%
\begin{Lemma}
  \label{thm:orbit-sum}
  \cite[Lemma~43]{Radu:RamanujanKolberg:2015}
  \begin{gather}
      \sum_{t' \in \modularOrbit{r,m,t}} (24 t'+ \sumdelta{r}) \equiv_m 0.
    \label{eq:orbit-rum}
  \end{gather}
\end{Lemma}

%%%%%%%%%%%%%%%%%%%%%%%%%%%%%%%%%%%%%%%%%%%%%%%%%%%%%%%%%%%%%%%%%%%
\begin{Lemma}
  \label{thm:stable-O-r-m-t}
  \label{thm:chi-exponent}
  \cite[Lemma~43]{Radu:RamanujanKolberg:2015}
  Let $t\in\Set{0,\ldots,m-1}$, $r \in R(M)$. Then
  \begin{enumerate}
  \item for any $x \in \setS_{24m}$ we have
    $\modularOrbit{r,m,t}
    = x \mathbin{\bar{\odot}_r} \modularOrbit{r,m,t}
    = \SetDef{x \mathbin{\bar{\odot}_r} t'}{t'\in \modularOrbit{r,m,t}}$,
  \item $\chi_{r,m,t}=\unityPower{\frac{\nu}{24}}$ for some
    $\nu \in \Set{0,\ldots,23}$.
  \end{enumerate}
\end{Lemma}

%%%%%%%%%%%%%%%%%%%%%%%%%%%%%%%%%%%%%%%%%%%%%%%%%%%%%%%%%%%%%%%%%%%
\begin{Definition}\cite[Def.~44]{Radu:RamanujanKolberg:2015}
  Given $t\in\Set{0,\ldots,m-1}$, $r \in R(M)$, $s \in R(N)$, we
  define
  \begin{align}
    \defineNotation[P-r-m-t-tau]{P_{r, m, t}(\tau)}
    &:= \prod_{k \in \modularOrbit{r,m,t}} g_{r,m,k}(\tau)\\
    %
    \defineNotation[F-s-r-m-t-tau]{F_{s, r, m, t}(\tau)}
    &:= \divisorprod{N} \eta(\delta\tau)^{s_\delta} \cdot P_{r, m, t}(\tau)
  \end{align}
\end{Definition}

%%%%%%%%%%%%%%%%%%%%%%%%%%%%%%%%%%%%%%%%%%%%%%%%%%%%%%%%%%%%%%%%%%%
By \eqref{eq:g_r-m-t(tau)} we can write the following.
\begin{align}
  P_{r, m, t}(\tau)
  &:= \prod_{k \in \modularOrbit{r,m,t}}
    \unityPower{\frac{\tau (24k+\sumdelta{r})}{24m}} (U_mf_{k})(\tau)\\
  &:= \prod_{k \in \modularOrbit{r,m,t}}
    \left[\unityPower{\frac{\tau (24k+\sumdelta{r})}{24m}}
    \cdot \sum_{n=0}^\infty a(mn+k) q^n\right]
  \\
  &:=
    \unityPower{\tau\sum_{k\in\modularOrbit{r,m,t}}\frac{24k+\sumdelta{r}}{24m}}
    \prod_{k \in \modularOrbit{r,m,t}} \left[\sum_{n=0}^\infty a(mn+k) q^n\right]
  \\
  %
  F_{s, r, m, t}(\tau)
  &:= \divisorprod{N} \eta(\delta\tau)^{s_\delta} \cdot P_{r,m,t}(\tau)
  \\
  &= \divisorprod{N} \eulerFunction{\delta}^{s_\delta}
    \cdot
    \unityPower{\tau\frac{\sumdelta{s}}{24}}
    \cdot
    \unityPower{\tau\sum_{k\in\modularOrbit{r,m,t}}\frac{24k+\sumdelta{r}}{24m}}
    \cdot
    \prod_{k \in \modularOrbit{r,m,t}} \left(\sum_{n=0}^\infty a(mn+k) q^n\right)
    \label{eq:F_s-r-m-t(tau)}
  \\
  &= \divisorprod{N} \eulerFunction{\delta}^{s_\delta}
    \cdot
    % \unityPower{\tau
    q^{
    \left(
    \frac{\sumdelta{s}}{24}
    +
    \sum_{k\in\modularOrbit{r,m,t}}\frac{24k+\sumdelta{r}}{24m}
    \right)
    }
    \cdot
    \prod_{k \in \modularOrbit{r,m,t}} \left(\sum_{n=0}^\infty a(mn+k) q^n\right)
  \\
  &= \divisorprod{N} \eulerFunction{\delta}^{s_\delta}
    \cdot
    % \unityPower{\tau
    q^\alpha
    \cdot
    \prod_{k \in \modularOrbit{r,m,t}} \left(\sum_{n=0}^\infty a(mn+k) q^n\right)
  \\
  \alpha
  &=
    \frac{\sumdelta{s}}{24}
    +
    \sum_{k\in\modularOrbit{r,m,t}}\frac{24k+\sumdelta{r}}{24m}
    \label{eq:alphaInfinity}
\end{align}

%%%%%%%%%%%%%%%%%%%%%%%%%%%%%%%%%%%%%%%%%%%%%%%%%%%%%%%%%%%%%%%%%%%
\subsubsection{Transformation of $P_{r,m,t}$ under $\SL2Z$}
%%%%%%%%%%%%%%%%%%%%%%%%%%%%%%%%%%%%%%%%%%%%%%%%%%%%%%%%%%%%%%%%%%%

\begin{align}
  \defineNotation[P-r-m-t-gamma-tau]{P_{r,m,t}(\gamma\tau)}
  &=
    \left(\frac{(c\tau+d)^{\divisorsum{M} \frac{r_\delta}{2}}}{m}\right)^\mu
    \prod_{k\in \modularOrbit{r,m,t}}
    \left[\sum_{\lambda=0}^{m-1}
    \unityPower{-\frac{\lambda}{24m}\Bigl(24k+\sumdelta{r}\Bigr)
    + \divisorsum{M}\frac{r_\delta(v_{\delta,m\lambda} +
    \kappa_{\gamma_{\delta,m,\lambda}})}{24}} \times\right.\notag\\
  &\qquad\times
    \left.
    \divisorprod{M}
    \left(\frac{h_{\delta,m,\lambda}}{\delta}\right)^{\!\frac{r_\delta}{2}}
    \cdot
    \unityPower{\frac{\divisorsum{M} r_\delta u_{\delta,m,\lambda}}{24}\tau}
    \cdot
    \divisorprod{M} \prod_{n=1}^\infty (1-q_{\delta,m,\lambda}^n)^{r_\delta}
    \right]
    \label{eq:P_r-m-t(gamma*tau)}
\end{align}

The parts of \eqref{eq:P_r-m-t(gamma*tau)} are implemented via
\textcolor{blue}{\code{SymbolicSiftedEtaQuotientOrbitProductGamma}}
(Section~\ref{sec:SymbolicSiftedEtaQuotientOrbitProductGamma}).

%%%%%%%%%%%%%%%%%%%%%%%%%%%%%%%%%%%%%%%%%%%%%%%%%%%%%%%%%%%%%%%%%%%
\subsubsection{Transformation of $F_{s,r,m,t}$ under $\SL2Z$}
%%%%%%%%%%%%%%%%%%%%%%%%%%%%%%%%%%%%%%%%%%%%%%%%%%%%%%%%%%%%%%%%%%%

%%%%%%%%%%%%%%%%%%%%%%%%%%%%%%%%%%%%%%%%%%%%%%%%%%%%%%%%%%%%%%%%%%%
\begin{Theorem}\cite[Thm.~45]{Radu:RamanujanKolberg:2015}
  \label{thm:RaduConditions}
  Let $(N, M, r, m, t) \in \Delta^*$ according to
  Definition~\ref{def:condition-co-eta-quotient-gamma0}, $s\in R(N)$,
  $\nu$ and $\mu$ be integers such that
  $\chi_{s,m,t}=\unityPower{\frac{\nu}{24}}$, and
  $\mu=\sizeOfSet{\modularOrbit{r,m,t}}$, then $F_{s,r,m,t}(\tau)$ is
  a modular function on $\Gamma_0(N)$ iff the following conditions
  hold:
  \begin{align}
    \divisorsum{N} s_\delta + \mu \divisorsum{M} r_\delta
    &=
      0,
    \label{eq:Radu-sum=0}\\
    \divisorsum{M} \delta s_\delta + \mu m \divisorsum{N} \delta r_\delta + \nu
    &\equiv
      0 \pmod{24},
    \label{eq:Radu-sigmainfinity}\\
    \divisorsum{N} \frac{N}{\delta} s_\delta
    + \frac{\mu m N}{M} \divisorsum{M} \frac{M}{\delta} r_\delta
    &\equiv
      0 \pmod{24},
    \label{eq:Radu-sigma0}\\
    \sqrt{
    \divisorprod{N}\delta^{s_\delta}
    \divisorprod{M} (m \delta)^{\mu r_\delta}
    } &\in \setQ\label{eq:Radu-productsquare}
  \end{align}
\end{Theorem}

%%%%%%%%%%%%%%%%%%%%%%%%%%%%%%%%%%%%%%%%%%%%%%%%%%%%%%%%%%%%%%%%%%%
Let $N, s, M, r, m, t, \nu, \mu$ fulfil all the conditions of
Theorem~\ref{thm:RaduConditions}. Then the exponent of the factor
$c\tau+d$ is zero, \ie, this factor vanishes.

\begin{align}
  \defineNotation[F-s-r-m-t-gamma-tau]{F_{s,r,m,t}(\gamma\tau)}
  &:= \divisorprod{N} \eta(\delta\gamma\tau)^{s_\delta}
    \cdot
    P_{r, m, t}(\gamma\tau)
  \notag\\
  &=
  (c\tau+d)^{\divisorsum{N} \frac{s_\delta}{2}}
  \cdot
  \divisorprod{N}
  \left(\frac{h_\delta}{\delta}\right)^{\!\frac{s_\delta}{2}}
  \cdot
  \unityPower{\divisorsum{N} \frac{s_\delta(v_\delta + \kappa_\delta)}{24}}
  \cdot
  q^{\divisorsum{N} \frac{s_\delta u_\delta}{24}}
  \cdot
    \divisorprod{N} \prod_{n=1}^{\infty}(1-q_\delta^n)^{s_\delta} \times\notag\\
  &\qquad\times
    P_{r, m, t}(\gamma\tau)
  \notag\\
  &=
  \divisorprod{N}
  \left(\frac{h_\delta}{\delta}\right)^{\!\frac{s_\delta}{2}}
  \cdot
  \unityPower{\divisorsum{N} \frac{s_\delta(v_\delta + \kappa_\delta)}{24}}
  \cdot
  q^{\divisorsum{N} \frac{s_\delta u_\delta}{24}}
    \cdot
    \divisorprod{N} \prod_{n=1}^{\infty}(1-q_\delta^n)^{s_\delta} \times\notag\\
  &\qquad\times
    \frac{1}{m^\mu}
    \prod_{k\in \modularOrbit{r,m,t}}
    \left[\sum_{\lambda=0}^{m-1}
    \unityPower{-\frac{\lambda}{24m}\Bigl(24k+\sumdelta{r}\Bigr)
    + \divisorsum{M}\frac{s_\delta(v_{\delta,m\lambda} +
    \kappa_{\gamma_{\delta,m,\lambda}})}{24}} \times\right.\notag\\
  &\qquad\times
    \left.
    \divisorprod{M}
    \left(\frac{h_{\delta,m,\lambda}}{\delta}\right)^{\!\frac{r_\delta}{2}}
    \cdot
    \unityPower{\frac{\divisorsum{M} r_\delta u_{\delta,m,\lambda}}{24}\tau}
    \cdot
    \divisorprod{M} \prod_{n=1}^\infty (1-q_{\delta,m,\lambda}^n)^{r_\delta}
    \right]
    \label{eq:F_s-r-m-t(gamma*tau)}
\end{align}


The parts of \eqref{eq:F_s-r-m-t(gamma*tau)} are implemented via
\textcolor{blue}{\code{SymbolicModularGamma0SiftedEtaQuotientGamma}}
(Section~\ref{sec:SymbolicModularGamma0SiftedEtaQuotientGamma}).












%%%%%%%%%%%%%%%%%%%%%%%%%%%%%%%%%%%%%%%%%%%%%%%%%%%%%%%%%%%%%%%%%%%
\subsubsection{The formulas in short and their relations to the
  implementation}
\label{sec:formulas-in-short}
%%%%%%%%%%%%%%%%%%%%%%%%%%%%%%%%%%%%%%%%%%%%%%%%%%%%%%%%%%%%%%%%%%%
$t \in \Set{0,1,\ldots,m-1}$

\begin{align*}
%%%%%%%%%%%%%%%%%%%%%%%%%%%%%%%%%%%%%%%%%%%%%%%%%%%%%%%%%%%%%%%%%%%
  f(\tau)
  &=
  \sum_{n=0}^\infty a(n) q^n
  =
  \divisorprod{M}\prod_{n=1}^\infty(1-q^{\delta n})^{r_\delta}
  =
  \divisorprod{M}\eulerFunction{\delta}^{r_\delta}
  =
  \unityPower{-\frac{\sumdelta{r} \tau}{24}} g_r(\tau)
  \\
%%%%%%%%%%%%%%%%%%%%%%%%%%%%%%%%%%%%%%%%%%%%%%%%%%%%%%%%%%%%%%%%%%%
  f_t(\tau)
  &:=
  \sum_{n=0}^\infty a(n) q^{n-t}
  =
  \sum_{n=-t}^\infty a(n+t) q^{n}
  =
  \unityPower{-t\tau}f(\tau)
  =
  \unityPower{-\frac{(24t+\sumdelta{r}) \tau}{24}} g_r(\tau)
  \\
%%%%%%%%%%%%%%%%%%%%%%%%%%%%%%%%%%%%%%%%%%%%%%%%%%%%%%%%%%%%%%%%%%%
  (U_mf_t)(\tau)
  &=
    \sum_{n=0}^\infty a(mn+t) q^n
  \\
  &=
  \frac{1}{m}
  \unityPower{-\frac{\tau (24t+\sumdelta{r})}{24m}}
  \sum_{\lambda=0}^{m-1}
  \unityPower{-\frac{\lambda (24t+\sumdelta{r})}{24m}}
  \divisorprod{M}
    \eta\left(\frac{\delta(\tau+\lambda)}{m}\right)^{r_\delta}
  \\
  &=
  \frac{1}{m}
  \unityPower{-\frac{\tau (24t+\sumdelta{r})}{24m}}
    \sum_{\lambda=0}^{m-1} g_{r,m,t,\lambda}(\tau)
\end{align*}
%%%%%%%%%%%%%%%%%%%%%%%%%%%%%%%%%%%%%%%%%%%%%%%%%%%%%%%%%%%%%%%%%%%

\textcolor{blue}{\texttt{SymbolicEtaQuotientLambdaGamma}}
(Section~\ref{sec:SymbolicEtaQuotientLambdaGamma})

\begin{align*}
  g_{r,m,t,\lambda}(\tau)
  &=
  \unityPower{-\frac{\lambda}{24m} (24t+\sumdelta{r})}
    \divisorprod{M}
    \eta\left(\frac{\delta(\tau+\lambda)}{m}\right)^{r_\delta}
  \\
  e &= \mathtt{etaQuotient}(M, \Delta, r, m, t, \lambda , \gamma)
  \\
  \mathtt{rationalPrefactor}(e)
  &=
    \divisorprod{M} \left(\frac{h_\delta}{\delta}\right)^{\!r_\delta}
  \\
  \mathtt{unityPower}(e)
  &=
    -\frac{\lambda}{24m}\Bigl(24t+\sumdelta{r}\Bigr) +
    \divisorsum{M} \frac{r_\delta(v_{\delta,m,\lambda} +
                         \kappa_{\gamma_{\delta,m,\lambda}})}{24}
  \\
  \mathtt{qExponent}(e)
  &=
    \divisorsum{M} \frac{r_\delta u_{\delta,m,\lambda}}{24}.
\end{align*}

%%%%%%%%%%%%%%%%%%%%%%%%%%%%%%%%%%%%%%%%%%%%%%%%%%%%%%%%%%%%%%%%%%%
\textcolor{blue}{\texttt{SymbolicSiftedEtaQuotientGamma}}
(Section~\ref{sec:SymbolicSiftedEtaQuotientGamma})

\begin{align*}
  g_{r,m,t}(\tau)
  &=
    \unityPower{\frac{\tau (24t+\sumdelta{r})}{24m} } (U_mf_t)(\tau)
    =
    \frac{1}{m} \sum_{\lambda=0}^{m-1} g_{r,m,t,\lambda}(\tau)
    =
    \unityPower{\frac{\tau (24t+\sumdelta{r})}{24m}}
    \sum_{n=0}^\infty a(mn+t) q^n
\end{align*}
%%%%%%%%%%%%%%%%%%%%%%%%%%%%%%%%%%%%%%%%%%%%%%%%%%%%%%%%%%%%%%%%%%%
\textcolor{blue}{\texttt{SymbolicSiftedEtaQuotientOrbitProductGamma}}
(Section~\ref{sec:SymbolicSiftedEtaQuotientOrbitProductGamma})

\begin{align*}
  P_{r, m, t}(\tau)
  &:=
    \frac{1}{m^\mu}
    \prod_{t' \in \modularOrbit{r,m,t}}
    \left(
    \sum_{\lambda=0}^{m-1} g_{r,m,t'}(\tau)
    \right)
\end{align*}
%%%%%%%%%%%%%%%%%%%%%%%%%%%%%%%%%%%%%%%%%%%%%%%%%%%%%%%%%%%%%%%%%%%
\textcolor{blue}{\texttt{SymbolicModularGamma0SiftedEtaQuotientGamma}}
(Section~\ref{sec:SymbolicSiftedEtaQuotientOrbitProductGamma})

\begin{align*}
  F_{s, r, m, t}(\tau)
  &:=
    \divisorprod{N} \eta(\delta \tau)^{s_\delta} \cdot P_{r, m, t}(\tau)
\end{align*}
%%%%%%%%%%%%%%%%%%%%%%%%%%%%%%%%%%%%%%%%%%%%%%%%%%%%%%%%%%%%%%%%%%%
\textcolor{blue}{\texttt{SymbolicEtaQuotientLambdaGamma}}
(Section~\ref{sec:SymbolicEtaQuotientLambdaGamma})
(actually unused)

\begin{align*}
  g_{r,m,-,\lambda}(\tau)
  &=
    \divisorprod{M}
    \eta\left(\frac{\delta(\tau+\lambda)}{m}\right)^{r_\delta}
\end{align*}
%%%%%%%%%%%%%%%%%%%%%%%%%%%%%%%%%%%%%%%%%%%%%%%%%%%%%%%%%%%%%%%%%%%
\textcolor{blue}{\texttt{SymbolicSiftedEtaQuotientGamma}}
(Section~\ref{sec:SymbolicSiftedEtaQuotientGamma})
(actually unused)

\begin{align*}
  g_{r,m}(\tau)
  &=
    (U_m g_r)(\tau)
    =
    \frac{1}{m} \sum_{\lambda=0}^{m-1} g_{r,m,-,\lambda}(\tau)
\end{align*}
%%%%%%%%%%%%%%%%%%%%%%%%%%%%%%%%%%%%%%%%%%%%%%%%%%%%%%%%%%%%%%%%%%%




























%%%%%%%%%%%%%%%%%%%%%%%%%%%%%%%%%%%%%%%%%%%%%%%%%%%%%%%%%%%%%%%%%%%
\subsection{Application of $U_m$ to $g_{r}$}
%%%%%%%%%%%%%%%%%%%%%%%%%%%%%%%%%%%%%%%%%%%%%%%%%%%%%%%%%%%%%%%%%%%

Let $m \in \setN$, $m>0$, $r \in R^*(mN)$,
$\defineNotation[gamma]{\gamma} =
\left(\begin{smallmatrix}a&b\\c&d\end{smallmatrix}\right) \in
\SL2Z$.
%

In this section we want to determine a formula for
$g_{r,m} := (U_m g_r)$.

We determine the expansions of $(U_m g_r)(\gamma\tau)$, similar to
Section~\ref{sec:function-g_r-m-t}.
%
In fact, we can reuse everything from
Section~\ref{sec:transformation-eta_delta-m-lambda} and then because
of $t=0$, $\divisorsum{M} r_\delta=0$ and the missing factor
$\unityPower{-\frac{\tau(24t+\sumdelta{r})}{24}}$
equation~\eqref{eq:g_r-m-t-lambda(gamma*tau)} simply turns into the
following.
\begin{align}
  g_{r,m,-,\lambda}(\gamma\tau)
  &=
    \divisorprod{M}
    \eta_{\delta,m,\lambda}(\gamma\tau)^{r_\delta}\notag\\
  \begin{split}
  &=
  \unityPower{
      \divisorsum{M}\frac{r_\delta(v_{\delta,m,\lambda}
        +
      \kappa_{\gamma_{\delta,m,\lambda}})}{24}} \times\\
  & \qquad\times
    \divisorprod{M}
    \left(\frac{h_{\delta,m,\lambda}}{\delta}\right)^{\!\frac{r_\delta}{2}}
    \cdot
    \unityPower{\frac{\divisorsum{M} r_\delta u_{\delta,m,\lambda}}{24} \tau}
    \cdot
    \divisorprod{M}
    \prod_{n=1}^\infty (1-q_{\delta,m,\lambda}^n)^{r_\delta}
  \end{split}
  \label{eq:g_r-m---lambda(gamma*tau)}
\end{align}

Then
\begin{align}
  g_{r,m}(\gamma\tau)
  &:=\frac{1}{m} \sum_{\lambda=0}^{m-1} g_{r,m,-,\lambda}(\gamma\tau)\notag\\
  \begin{split}
  &=
    \sum_{\lambda=0}^{m-1}
    \unityPower{\divisorsum{M}\frac{r_\delta(v_{\delta,m\lambda} +
    \kappa_{\gamma_{\delta,m,\lambda}})}{24}} \times\\
  & \qquad\times
    \divisorprod{M}
    \left(\frac{h_{\delta,m,\lambda}}{\delta}\right)^{\!\frac{r_\delta}{2}}
    \cdot
    \unityPower{\frac{\divisorsum{M} r_\delta u_{\delta,m,\lambda}}{24}\tau}
    \cdot
    \divisorprod{M}
    \prod_{n=1}^\infty (1-q_{\delta,m,\lambda}^n)^{r_\delta}
  \end{split}
  \label{eq:g_r-m(gamma*tau)}
\end{align}


%%%%%%%%%%%%%%%%%%%%%%%%%%%%%%%%%%%%%%%%%%%%%%%%%%%%%%%%%%%%%%%%%%%
 \subsubsection{The trace and the Atkin-Lehner involution applied to
   $g_r$}
%%%%%%%%%%%%%%%%%%%%%%%%%%%%%%%%%%%%%%%%%%%%%%%%%%%%%%%%%%%%%%%%%%%

See, for example, \cite{Kohnen:WeierstrassPointsAtInfinity:2004}.

We only treat the special case where $m=2$ and $N=121$.

Then
\begin{align*}
  W = W_2^{242}
  &=
    \begin{pmatrix}
      2 & 0\\
      242 & 1
    \end{pmatrix}\\
    %
  \gamma_{2,0}
  &=
    \begin{pmatrix}
      2 & 0\\
      242 & 1
    \end{pmatrix}
    \begin{pmatrix}
      1 & 0\\
      0 & 2
    \end{pmatrix}
  =
    \begin{pmatrix}2&0\\242&2\end{pmatrix}
  =
    \begin{pmatrix}1&0\\121&1\end{pmatrix}
    \begin{pmatrix}2&0\\0&2\end{pmatrix}
  \\
  \gamma_{2,1}
  &=
    \begin{pmatrix}
      2 & 0\\
      242 & 1
    \end{pmatrix}
    \begin{pmatrix}
      1 & 1\\
      0 & 2
    \end{pmatrix}
  =
    \begin{pmatrix}2&2\\242&244\end{pmatrix}
  =
    \begin{pmatrix}1&1\\121&122\end{pmatrix}
    \begin{pmatrix}2&0\\0&2\end{pmatrix}
\end{align*}

Since $\left(\begin{smallmatrix}2&0\\0&2\end{smallmatrix}\right)\tau =
\tau$, we can simply ignore this matrix and write
\begin{align*}
  (U_2 g_{r, W})(\tau)
  &=
    \frac{1}{2}\left(g_r(\gamma'_{2,0}\tau) + g_r(\gamma'_{2,1}\tau)\right)
\end{align*}
for
$\gamma'_{2,0} =
\left(\begin{smallmatrix}1&0\\121&1\end{smallmatrix}\right)$ and
$\gamma'_{2,0} =
\left(\begin{smallmatrix}1&1\\121&122\end{smallmatrix}\right)$.

Since 2 is a prime, we get from
\cite{Kohnen:WeierstrassPointsAtInfinity:2004} that
$\trace_{121}^{242}: M^\infty(242) \to M^\infty(121)$ can be computed via
\begin{gather*}
  g_r|\trace_{121}^{242}
  = g_r + 2 U_2(g_r|W_2^{242})
  = g_r + g_r|\gamma'_{2,0} + g_r|\gamma'_{2,1}.
\end{gather*}
Note that the width of $\gamma'_{2,0}$ and $\gamma'_{2,1}$ is 2, so in
our formula \eqref{eq:modular-g_r(gamma*tau)} we have
$x=q^{\frac{1}{2}}$.

Indeed the identity matrix, $\gamma'_{2,0}$, and $\gamma'_{2,1}$ are
three coset representative for
$\Gamma_0(242) \backslash \Gamma_0(121)$.
%
See function \code{rightCosetRepresentatives} in
\PathName{qetaauxmeq.spad} and \cite[Lemma~2.44]{Radu:PhD:2010}.































%%%%%%%%%%%%%%%%%%%%%%%%%%%%%%%%%%%%%%%%%%%%%%%%%%%%%%%%%%%%%%%%%%%
\subsection{Related to Generalized Eta-Quotients}
%%%%%%%%%%%%%%%%%%%%%%%%%%%%%%%%%%%%%%%%%%%%%%%%%%%%%%%%%%%%%%%%%%%


%%%%%%%%%%%%%%%%%%%%%%%%%%%%%%%%%%%%%%%%%%%%%%%%%%%%%%%%%%%%%%%%%%%
\subsubsection{Definition of Generalized Eta-Functions}
%%%%%%%%%%%%%%%%%%%%%%%%%%%%%%%%%%%%%%%%%%%%%%%%%%%%%%%%%%%%%%%%%%%

\begin{Hemmecke}
  The transformation rules for the generalized eta-function are
  usually given in terms of their notation by Schoeneberg, see
  \cite[Chp.~VIII]{Schoeneberg:EllipticModularFunctions:1974} and
  \cite{Chen+Du+Zhao:FindingModularFunctionsRamanujan:2019}. For our
  purpose we add additional indices $h$ and $N$ to refer to the
  implicit parameters used by Schoeneberg. We also add an upper index
  $[S]$ to refer to Schoeneberg's definition.
\end{Hemmecke}

In the following we always use $q = e^{2\pi i \tau}$ for $\tau \in
\setH$.
%
Furthermore let $N$ be a positive integer and $\zeta_N$ a primitive
$N$-th root of unity.

Let $\defineNotation[B]{B(x)} = x^2 - x + \frac{1}{6}$
be the second Bernoulli function, $\{x\}$.
%
Denote the fractional part of $x$, by
$\defineNotation[x-fractional]{\{x\}} = x - \lfloor x \rfloor$,
$\defineNotation[x-floor]{\lfloor x \rfloor}$ denotes the greatest
integer less than or equal to $x$.
We define
\begin{gather}
    \defineNotation[P-2]{P_2(x)} = B(\{x\}) = \{x\}^2 - \{x\} + \frac{1}{6}.
\end{gather}


%%%%%%%%%%%%%%%%%%%%%%%%%%%%%%%%%%%%%%%%%%%%%%%%%%%%%%%%%%%%%%%%%%%
\begin{Definition}
  (\cite{Schoeneberg:EllipticModularFunctions:1974},
  \cite[p.~672]{Yang:GeneralizedDedekindEtaFunctions:2004})
  Let $g$ and $h$ be
  real numbers, then the generalized eta-function $\eta_{g,h}^{[S]}$
  is defined by
\begin{align}
  \defineNotation[eta-g-h-tau-Schoeneberg]{\eta_{g,h}^{[S]}(\tau)}
  &:=
  \alpha(g,h,N) q^{\frac{1}{2} P_2(\frac{g}{N})}
  \prod_{\substack{n>0\\n\equiv g \bmod N}}
       (1-\zeta_N^h q^{n/N})
  \prod_{\substack{n>0\\n\equiv -g \bmod N}}
       (1-\zeta_N^{-h} q^{n/N})
  \notag\\
  &=
  \alpha(g,h,N) q^{\frac{1}{2} P_2(\frac{g}{N})}
  \prod_{n=1}^\infty
    \bigl(1-\zeta_N^h q^{n-1 + \left\{\frac{g}{N}\right\}}\bigr)
    \bigl(1-\zeta_N^{-h} q^{n - \left\{\frac{g}{N}\right\}}\bigr)
  \notag\\
  &=
  \alpha(g,h,N) q^{\frac{1}{2} P_2(\frac{g}{N})}
  \left( \zeta_N^h q^{\left\{\frac{g}{N}\right\}}; q \right)
  \left( \zeta_N^{-h} q^{1-\left\{\frac{g}{N}\right\}}; q \right)
  \label{eq:eta_g-h^[S](tau)}
\end{align}
where $0\le\left\{\frac{g}{N}\right\}<N$ is the fractional part of
$\frac{g}{N}$,
\begin{gather}
  \defineNotation[alpha-g-h-N]{\alpha(g,h,N)}
  :=
  \begin{cases}
    (1 - \zeta_N^{-h}) \unityPower{\frac{1}{2} P_1\!\left(\frac{h}{N}\right)},
    &
    \text{if $g\equiv0\pmod{N}$ and $h\not\equiv0\pmod{N}$,}
    \\
    1, & \text{otherwise},
  \end{cases}
  \label{eq:eta_g-h^[S](tau)-prefactor}
\end{gather}
and $P_1(x)$ is the first Bernoulli function given by
\begin{gather*}
  \defineNotation[P-1]{P_1(x)}
  =
  \{x\} - \frac{1}{2}.
\end{gather*}

Since there is an implicit parameter $N$, we sometimes add an explicit
third index $N$, \ie
\begin{align}
  \label{eq:eta_g-h-N^[S](tau)}
  \defineNotation[eta-g-h-N-tau-Schoeneberg]{\eta_{g,h,N}^{[S]}(\tau)}
  &=
  \eta_{g,h}^{[S]}(\tau),
\end{align}
in order to make this parameter explicit.
\end{Definition}
%%%%%%%%%%%%%%%%%%%%%%%%%%%%%%%%%%%%%%%%%%%%%%%%%%%%%%%%%%%%%%%%%%%

Obviously, $\eta_{g,h}^{[S]}(\tau) = \eta_{g+uN,h+vN}^{[S]}(\tau)$ for
any integers $u$ and $v$.

Another definition can be found in
\cite{Robins:GeneralizedDedekindEtaProducts:1994} and
\cite{Chen+Du+Zhao:FindingModularFunctionsRamanujan:2019}.


%%%%%%%%%%%%%%%%%%%%%%%%%%%%%%%%%%%%%%%%%%%%%%%%%%%%%%%%%%%%%%%%%%%
\begin{Definition}
\cite{Robins:GeneralizedDedekindEtaProducts:1994}
Let $\delta$ be a positive natural number and $0 \le g < \delta$ be a
residue class (mod $\delta$).
%
We consider the generalized eta-function which is given by
\begin{align}
  \defineNotation[eta-delta-g-tau-Robins]{\eta_{\delta,g}^{[R]}(\tau)}
  &:=
  q^{\frac{\delta}{2}P_2(\frac{g}{\delta})}
  %
  \prod_{\substack{n>0\\n\equiv g\ (\mathrm{mod}\ \delta)}} (1-q^n)
  \prod_{\substack{n>0\\n\equiv -g\ (\mathrm{mod}\ \delta)}} (1-q^n)
  \notag\\
  &=
  q^{\frac{\delta}{2}P_2(\frac{\bar{g}}{\delta})}
  %
  \prod_{n=1}^\infty (1-q^{\delta (n-1)+\bar{g}})(1-q^{\delta n-\bar{g}})
  \notag\\
  &=
  q^{\frac{\delta}{2}P_2(\frac{\bar{g}}{\delta})}
  %
  (q^{\bar{g}}; q^\delta) (q^{\delta-\bar{g}}; q^\delta)
  \label{eq:eta_delta-g^[R](tau)}
\end{align}
where $\bar{g}\in\Set{0,\ldots,\delta-1}$ is such that
$\bar{g} \equiv g \pmod{\delta}$,
\end{Definition}
%%%%%%%%%%%%%%%%%%%%%%%%%%%%%%%%%%%%%%%%%%%%%%%%%%%%%%%%%%%%%%%%%%%


Yet another definition is given by
\cite{Yang:GeneralizedDedekindEtaFunctions:2004}.


%%%%%%%%%%%%%%%%%%%%%%%%%%%%%%%%%%%%%%%%%%%%%%%%%%%%%%%%%%%%%%%%%%%
\begin{Definition}
  \cite{Yang:GeneralizedDedekindEtaFunctions:2004}
  %
  Let $g$ and $h$ be real numbers not simultaneously congruent to 0
  modulo $N$.
%
The generalized Dedekind eta-function is given by
\begin{align}
  \defineNotation[E-g-h-tau]{E_{g,h}(\tau)}
  &:=
  \defineNotation[eta-g-h-tau-Yang]{\eta_{g,h}^{[Y]}(\tau)}
  :=
  q^{\frac{1}{2} B(\frac{g}{N})}
  \prod_{n=1}^\infty
    \bigl(1-\zeta_N^h q^{n-1 + \frac{g}{N}}\bigr)
    \bigl(1-\zeta_N^{-h} q^{n - \frac{g}{N}}\bigr)
  \notag\\
  &=
  q^{\frac{1}{2} B(\frac{g}{N})}
    \left( \zeta_N^h q^{\frac{g}{N}}; q \right)
    \left( \zeta_N^{-h} q^{1-\frac{g}{N}}; q \right)
  \label{eq:eta_g-h^[Y](tau)}
\end{align}
For $g \not\equiv 0 \pmod{N}$ define
\begin{align}
  \defineNotation[E-g-tau]{E_g(\tau)}
  &:=
  \defineNotation[eta-g-tau-Yang]{\eta_g^{[Y]}(\tau)}
  :=
  E_{g,0}(N \tau)
  =
  q^{\frac{N}{2} B(\frac{g}{N})}
  \prod_{n=1}^\infty
    \bigl(1-q^{N (n-1) + g}\bigr)
    \bigl(1-q^{N n - g}\bigr).
  \notag\\
  &=
  q^{\frac{N}{2} B(\frac{g}{N})}
    \left( q^g; q \right)
    \left( q^{N-g}; q \right)
  \label{eq:eta_g^[Y](tau)}
\end{align}
\end{Definition}
%%%%%%%%%%%%%%%%%%%%%%%%%%%%%%%%%%%%%%%%%%%%%%%%%%%%%%%%%%%%%%%%%%%



%%%%%%%%%%%%%%%%%%%%%%%%%%%%%%%%%%%%%%%%%%%%%%%%%%%%%%%%%%%%%%%%%%%
\begin{Lemma}
  Let $k\in\setN$ be such that $N=k \delta$. Furthermore, let $g$ be a
  natural number with $0 < g < \delta$, and let $h$ be an arbitrary
  real number. Then the following holds.
  \begin{align}
    E_{g,h}(\tau) &= \eta_{g,h}^{[S]}(\tau)
    \\
    %
    E_g(\tau)
    &=
      E_{g,0}(N \tau)
      =
      \eta_{g,0}^{[S]}(N \tau)
      =
      \eta_{N,g}^{[R]}(\tau)
    \\
    %
    \eta_{\delta,g}^{[R]}(\tau)
    &=
    \prod_{i=0}^{k-1} E_{\delta i + g}(\tau)
  \end{align}
\end{Lemma}
%%%%%%%%%%%%%%%%%%%%%%%%%%%%%%%%%%%%%%%%%%%%%%%%%%%%%%%%%%%%%%%%%%%
\begin{proof}
  The first two lines are obvious from the definition.
  %
  The last one follows from
  \begin{gather*}
    \prod_{m=1}^\infty \bigl(1-q^{\delta m + g}\bigr) =
    % \prod_{\substack{n>0,0\le i<k\\m=kn+i}} \bigl(1-q^{\delta m+g}\bigr) =
    \prod_{n=1}^\infty \prod_{i=0}^{k-1} \bigl(1-q^{\delta (k n + i) + g}\bigr) =
    \prod_{i=0}^{k-1} \prod_{n=1}^\infty \bigl(1-q^{N n + \delta i + g}\bigr)
  \end{gather*}
  and
  \begin{gather*}
    \delta P_2\left(\frac{g}{\delta}\right)
    =
    \sum_{i=0}^{k-1}N B\left(\frac{\delta i + g}{N}\right).
  \end{gather*}
  Note that $0<g<\delta \le N$ and $B(x)=P_2(x)$ for $0<x<1$.
\end{proof}
%%%%%%%%%%%%%%%%%%%%%%%%%%%%%%%%%%%%%%%%%%%%%%%%%%%%%%%%%%%%%%%%%%%

The following Lemma is extracted from Theorem~1 and Corollary~2 of
\cite{Yang:GeneralizedDedekindEtaFunctions:2004} and says that we can
essentially restrict the indices to $0<g<N$ and $0\le h<N$.


%%%%%%%%%%%%%%%%%%%%%%%%%%%%%%%%%%%%%%%%%%%%%%%%%%%%%%%%%%%%%%%%%%%
\begin{Lemma}[Yang]
  \label{thm:modular-Yang-E}
  Let $g$ and $h$ be real numbers not simultaneously congruent to 0
  modulo $N$. Then
  \begin{gather}
    E_{g+N,h}(\tau) = E_{-g,-h}(\tau) = -\zeta_N^{-h} E_{g,h}(\tau)\\
    E_{g,h+N}(\tau) = E_{g, h}(\tau)
  \end{gather}
  Let $g \not\equiv 0 \pmod{N}$. Then
  \begin{gather}
    E_{g+N}(\tau) = E_{-g}(\tau) = - E_g(\tau)
  \end{gather}
\end{Lemma}
%%%%%%%%%%%%%%%%%%%%%%%%%%%%%%%%%%%%%%%%%%%%%%%%%%%%%%%%%%%%%%%%%%%

By simple comparison, we see that
\begin{gather}
  \eta_{g,h}^{[S]}(\tau)
  =
  \alpha(g,h,N) E_{g,h}(\tau)
  \label{eq:eta_g-h-Schoneberg-Yang}
\end{gather}
if $0<g<N$ and $0\le h<N$.


%%%%%%%%%%%%%%%%%%%%%%%%%%%%%%%%%%%%%%%%%%%%%%%%%%%%%%%%%%%%%%%%%%%
\subsubsection{Transformations of Generalized Eta-Functions}
%%%%%%%%%%%%%%%%%%%%%%%%%%%%%%%%%%%%%%%%%%%%%%%%%%%%%%%%%%%%%%%%%%%

Since $\eta_{\delta,0}^{[R]}(\tau) = \eta(\delta\tau)^2$ and we already have
a transformation formula for the pure eta-function given by
\eqref{eq:eta_delta(gamma*tau)}, we only need to consider
transformation rules with $g \not\equiv 0 \pmod{\delta}$.

A transformation formula was given in
\cite[Chp.~VIII]{Schoeneberg:EllipticModularFunctions:1974}, see
also \cite[p.~673]{Yang:GeneralizedDedekindEtaFunctions:2004}.

%%%%%%%%%%%%%%%%%%%%%%%%%%%%%%%%%%%%%%%%%%%%%%%%%%%%%%%%%%%%%%%%%%%
\begin{Lemma}
  \label{thm:Generalized-Eta-Transformation-Schoeneberg}
  Let
  $\left(\begin{smallmatrix}a&b\\c&d\end{smallmatrix}\right) \in
  \SL2Z$, $N>0$ be an integer, and $g$, $h$ two real numbers not
  simultaneously congruent to 0 modulo $N$.
  %
  Then
  \begin{gather}
    \label{eq:eta_g-h-N^[S](gamma*tau)}
    \defineNotation[eta-g-h-N-gamma-tau-Schoeneberg]%
      {\eta_{g,h,N}^{[S]}(\gamma\tau)}
    =
    \unityPower{\kappa_{g,h,N,\gamma}^{[S]}} \, \eta_{g',h',N}^{[S]}(\tau)
  \end{gather}
  where $\unityPowerSymbol$ is defined in Definition~\ref{def:epsilon},
  \begin{gather}
    \defineNotation[kappa-g-h-N-gamma-Schoeneberg]{\kappa_{g,h,N,\gamma}^{[S]}}
    =
    \begin{cases}
      \frac{b}{2d} P_2(\frac{g}{N}), &\text{if $c=0$},
      \\
      \frac{a}{2c} P_2(\frac{g}{N})
        + \frac{d}{2c} P_2(\frac{g'}{N})
        - \sign c \cdot s_{g,h,N}(a,c),
      &
      \text{if $c\not=0$},
    \end{cases}
    \label{eq:kappa_g-h-N-gamma-Schoeneberg}
  \end{gather}
  $g'=ag+ch$, $h'=bg+dh$, and $s_{g,h,N}(a,c)$ is the
  \emph{generalized Dedekind sum} defined by
  \begin{gather}
    \defineNotation[s-g-h-N]{s_{g,h,N}}(a,c)
    =
    \sum_{k \bmod c}
    \Bigl(\Bigl( \frac{g + k N}{c N} \Bigr)\Bigr)
    \Bigl(\Bigl( \frac{g' + k a N}{c N} \Bigr)\Bigr),
    \label{eq:s_g-h-N}
  \end{gather}
  where the summation runs over a complete set of representatives mod
  $c$ and
  \begin{gather*}
    \defineNotation[x-P1-0]{((x))}
    =
    \begin{cases}
      P_1(x) & \text{if $x\notin\setZ$},
      \\
      0, & \text{if $x \in \setZ$},
    \end{cases}
  \end{gather*}
  see \cite[p.~673]{Yang:GeneralizedDedekindEtaFunctions:2004}.
\end{Lemma}
%%%%%%%%%%%%%%%%%%%%%%%%%%%%%%%%%%%%%%%%%%%%%%%%%%%%%%%%%%%%%%%%%%%

Note that because in \eqref{eq:s_g-h-N} the sum runs over all
representatives mod $c$, we have
\begin{gather*}
  s_{g,h,N}(a,c)=s_{g+N,h,N}(a,c)=s_{g,h+N,N}(a,c).
\end{gather*}



%%%%%%%%%%%%%%%%%%%%%%%%%%%%%%%%%%%%%%%%%%%%%%%%%%%%%%%%%%%%%%%%%%%
\begin{Lemma}[Yang]
  \cite[Theorem~1]{Yang:GeneralizedDedekindEtaFunctions:2004}.
  \label{thm:Generalized-Eta-Transformation-Yang}
  Let
  $\left(\begin{smallmatrix}a&b\\c&d\end{smallmatrix}\right) \in
  \SL2Z$, $N>0$ be an integer, and $g$, $h$ two real numbers not
  simultaneously congruent to 0 modulo $N$.
  %
  Then
  \begin{gather}
    E_{g,h}(\gamma\tau)
    =
    \unityPower{\kappa_{g,h,N,\gamma}^{[Y]}}E_{g',h'}(\tau),
  \end{gather}
  where $g'=ag+ch$, $h'=bg+dh$, and
  \begin{gather}
    \kappa_{g,h,N,\gamma}^{[Y]}
    =
    \begin{cases}
      \frac{b}{2d} B(\frac{g}{N}), &\text{if $c=0$},
      \\
      \frac{c (a+d-3) + b d (1 - c^2)}{12} + \frac{\delta}{2}
      &
      \text{if $c$ is odd}
      \\
      \frac{a c (1 - d^2) + d (b - c + 3)}{12} + \frac{\delta}{2} - \frac{1}{4}
      &
      \text{if $c\not=0$ and $d$ is odd}
    \end{cases}
  \end{gather}
  for
  \begin{gather}
    \delta = \frac{g^2 a b + 2 g h b c + h^2 c d}{N^2}
             - \frac{g b + h (d-1)}{N}.
  \end{gather}
\end{Lemma}
%%%%%%%%%%%%%%%%%%%%%%%%%%%%%%%%%%%%%%%%%%%%%%%%%%%%%%%%%%%%%%%%%%%

Clearly, according to Lemma~\ref{thm:modular-Yang-E}, for some integer
$z$ it must hold:
\begin{gather*}
  \kappa_{g,h,N,\gamma}^{[S]}
  =
  z + \kappa_{g,h,N,\gamma}^{[Y]}
    + \left\lfloor\frac{g'}{N} \right\rfloor
      \left(\frac{1}{2} - \frac{h'}{N}\right)
\end{gather*}
If $g\not\equiv0\pmod{0}$ and $g'\not\equiv0\pmod{0}$.



%%%%%%%%%%%%%%%%%%%%%%%%%%%%%%%%%%%%%%%%%%%%%%%%%%%%%%%%%%%%%%%%%%%
\subsubsection{Generalized Eta-Quotients that are Modular for $\Gamma_1(N)$}
%%%%%%%%%%%%%%%%%%%%%%%%%%%%%%%%%%%%%%%%%%%%%%%%%%%%%%%%%%%%%%%%%%%

In view of $\eta_{\delta,g}^{[R]}(\tau) = \eta_{\delta,\delta-g}^{[R]}(\tau)$ we
can say that a generalized eta-quotient is a product of the form
\begin{gather}
  \prod_{\substack{\divides{\delta}{N}\\ 0\le g \le \lfloor\delta/2\rfloor}}
       \eta_{\delta,g}^{[R]}(\tau)^{a_{\delta,g}}
\end{gather}
where
\begin{gather}
  \defineNotation[a-delta-g]{a_{\delta,g}}\in
  \begin{cases}
    \frac{1}{2}\setZ & \text{if $g=0$ or $g=\frac{\delta}{2}$},\\
    \setZ & \text{otherwise}.
  \end{cases}
\end{gather}
Since $\eta_{\delta,0}^{[R]}(\tau) = \eta(\delta\tau)^2$ and
$\eta_{\delta,\frac{\delta}{2}}^{[R]}(\tau)
=
\frac{\eta(\frac{\delta}{2}\tau)^2}{\eta(\delta\tau)^2},
$
we can remove some redundancy from
\eqref{eq:redundant-generalized-eta-quotient} by rewriting (in case of
$\divides{2}{\delta}$)
\begin{gather}
  \eta_{\delta,\frac{\delta}{2}}^{[R]}(\tau)
  =
  \frac{\eta_{\frac{\delta}{2}, 0}^{[R]}(\tau)}{\eta_{\delta,0}^{[R]}(\tau)}
\end{gather}
and thus represent a generalized eta-quotient by
\begin{gather}
  \prod_{\substack{\divides{\delta}{N}\\0\le g < \delta/2}}
       \eta_{\delta,g}^{[R]}(\tau)^{a_{\delta,g}}.
  \label{eq:redundant-generalized-eta-quotient}
\end{gather}

By $\eta_{\delta,0}^{[R]}(\tau) = \eta(\delta\tau)^2$ and if we set
$\defineNotation[r-delta-0]{r_{\delta,0}} = 2 a_{\delta,0}$ and
$\defineNotation[r-delta-g]{r_{\delta,g}}=a_{\delta,g}$ for
$0<g<\frac{\delta}{2}$, we can represent a generalized eta-quotient as
\begin{gather}
  \defineNotation[g-rbar-tau]{g_{\bar{r}}(\tau)} =
  \prod_{\delta|N} \eta(\delta\tau)^{r_{\delta,0}}
  \prod_{\substack{\divides{\delta}{N}\\0 < g < \frac{\delta}{2}}}
  \eta_{\delta,g}^{[R]}(\tau)^{r_{\delta,g}}
  \label{eq:g_rbar(tau)}
\end{gather}
where $ r_{\delta,g} \in\setZ$ for every
$\divides{\delta}{N}$ and $0 \le g< \frac{\delta}{2}$.
Above we denoted by $\bar{r}$ the integer tuple
\begin{gather*}
  \defineNotation[rbar]{\bar{r}}
  =
  (r_{\delta,g})_{\divides{\delta}{N}, 0\le g < \delta/2}.
\end{gather*}

\begin{Definition}
  \label{def:rbar}
  When
  \begin{gather}
    \defineNotation[rbar]{\bar{r}}
    =
    (r_{\delta,g})_{\divides{\delta}{N}, 0\le g < \delta/2}.
    \label{eq:rbar}
  \end{gather}
  is an integer tuple, indexed by the index pairs
  \begin{gather}
    (\delta_1,0), (\delta_2,0), \ldots, (\delta_n,0),
    (\delta_2,1),\ldots, \Bigl(\delta_2,
    \Bigl\lceil\frac{\delta_2}{2} \Bigr\rceil-1\Bigr),
    \ldots,
    (\delta_n,1),
    \ldots,
    \Bigl(\delta_n,\Bigl\lceil\frac{\delta_n}{2}\Bigr\rceil-1\Bigr)
    \label{eq:sorted-indices}
  \end{gather}
  where $\delta_1<\delta_2< \ldots < \delta_n$ are the positive
  divisors of $N$, we sometimes denote by $r$ the part of the tuple
  that has a zero in the second subindex, \ie,
  \begin{gather}
    r = (r_{\delta,0})_{\divides{\delta}{N}}
  \end{gather}
  By $\tilde{r}$ we denote the other part, \ie,
  \begin{gather}
    \defineNotation[rtilde]{\tilde{r}}
    =
    (r_{\delta,g})_{\divides{\delta}{N}, 0 < g < \delta/2}.
    \label{eq:rtilde}
  \end{gather}
  Note that $\delta_1=1$ and $\delta_n=N$. Therefore, $(\delta_1,1)$
  is missing from the above list, since 1 is not smaller than
  $\delta_1=1$.

  By $\defineNotation[RbarN]{\bar{R}(N)}$ we denote the tuples with
  integer entries given in \eqref{eq:rbar}, \ie, an element
  $\bar{r} \in \bar{R}(N)$ represents exponents of the eta-product
  $g_{\bar{r}}(\tau)$ given in \eqref{eq:g_rbar(tau)}.

  We someimes write $r_\delta$ instead of $r_{\delta,0}$, since it
  corresponds to the integer exponent of the (pure) eta-function
  $\eta(\delta\tau)$.

  According to \eqref{eq:sorted-indices} we have
  $\bar{r} = (r, \tilde{r})$.
\end{Definition}

According to \cite[Thm.~3]{Robins:GeneralizedDedekindEtaProducts:1994}
a generalized eta-quotient of the form
\eqref{eq:redundant-generalized-eta-quotient} is a modular function
for $\Gamma_1(N)$ if
\begin{gather}
  \sum_{\divides{\delta}{N}} a_{g,0} = 0\\
  \sum_{\substack{\divides{\delta}{N}\\0 \le g \le \frac{\delta}{2}}}
  \delta P_2\Bigl(\frac{g}{\delta}\Bigr) a_{\delta,g} \equiv 0 \pmod{2}\\
  \sum_{\substack{\divides{\delta}{N}\\0 \le g \le \frac{\delta}{2}}}
  \frac{N}{\delta} P_2(0) a_{\delta,g} \equiv 0 \pmod{2}.
\end{gather}

Translated for an eta-quotient $g_{\bar{r}}$ given by
\eqref{eq:g_rbar(tau)}, the above reads as
\begin{align}
  \sum_{\divides{\delta}{N}} r_{g,0} = 0,
  \label{eq:generalized-weight}
\end{align}
which means that the weight (of the eta-quotient) should be 0,
\begin{align}
  \sum_{\divides{\delta}{N}}
  \delta P_2(0) \frac{r_{\delta,0}}{2}
  +
  \sum_{\substack{\divides{\delta}{N}\\0 < g < \frac{\delta}{2}}}
  \delta P_2\Bigl(\frac{g}{\delta}\Bigr) r_{\delta,g}
  &\equiv 0 \pmod{2}    \notag\\
  %
  \sum_{\divides{\delta}{N}}
  \delta r_{\delta,0}
  +
  12 \sum_{\substack{\divides{\delta}{N}\\0 < g < \frac{\delta}{2}}}
  \delta P_2\Bigl(\frac{g}{\delta}\Bigr) r_{\delta,g}
  &\equiv 0 \pmod{24}
  \label{eq:generalized-sigmaInfinity}
\end{align}
and
\begin{align}
\sum_{\divides{\delta}{N}}
  \frac{N}{\delta} P_2(0) \frac{r_{\delta,g}}{2}
  +
  \sum_{\substack{\divides{\delta}{N}\\0 < g < \frac{\delta}{2}}}
  \frac{N}{\delta} P_2(0) r_{\delta,g} \equiv 0 \pmod{2}  \notag\\
  %
  \sum_{\divides{\delta}{N}}
  \frac{N}{\delta} r_{\delta,0}
  +
  2 \sum_{\substack{\divides{\delta}{N}\\0 < g < \frac{\delta}{2}}}
  \frac{N}{\delta} r_{\delta,g} \equiv 0 \pmod{24}.
  \label{eq:generalized-sigma0}
\end{align}

Compare \eqref{eq:generalized-sigmaInfinity} and
\eqref{eq:generalized-sigma0} with the implementation of
\code{sigmaInfinity} and \code{sigma0} above as well as
\code{conditionSigma0?} in \code{QEtaCoFactorSpace}.


Similar to the matrix by Ligozat, we can build a matrix to determine
the order of a (modular for $\Gamma_1(N)$) generallized eta-quotient
at all cusps of $\Gamma_1(N)$.


Let us state a result of Robins about the order of an eta-quotient at
a cusp of $\Gamma_1(N)$.

%%%%%%%%%%%%%%%%%%%%%%%%%%%%%%%%%%%%%%%%%%%%%%%%%%%%%%%%%%%%%%%%%%%
\begin{Theorem}
  \cite[Thm~4]{Robins:GeneralizedDedekindEtaProducts:1994}
  %
  Let $\bar{r}\in \bar{R}(N)$ and $g_{\bar{r}}(\tau)$ be an
  eta-quotient as specified in \eqref{eq:g_rbar(tau)}.
  Let
  $\gamma=\left(\begin{smallmatrix}a&b\\c&d\end{smallmatrix}\right)\in\SL2Z$.
  Let
  $\lambda,\mu,\epsilon$ with $\divides{\epsilon}{N}$ and
  \begin{gather}
    \gcd(\lambda,N)=\gcd(\lambda,\mu)=\gcd(\mu,N)=1
  \end{gather}
  be such that
  $\frac{\lambda}{\mu\epsilon}$ is a cusp that is
  $\Gamma_1(N)$-equivalent to the cusp $\frac{a}{c}$ of $\Gamma_1(N)$.
  %
  Then the order of the transformation $g_{\bar{r}}(\gamma\tau)$ in
  the uniformizing variable $q^{\epsilon/N}$ is
  \begin{gather}
    \frac{N}{24}  \sum_{\divides{\delta}{N}}
    \frac{\gcd(\delta,\epsilon)^2}{\delta\epsilon} r_\delta
    +
    \frac{N}{2}  \sum_{\substack{\divides{\delta}{N}\\0 < g < \frac{\delta}{2}}}
    \frac{\gcd(\delta,\epsilon)^2}{\delta\epsilon}
    P_2\Bigl(\frac{\lambda g}{\gcd(\delta,\epsilon)}\Bigr)r_{\delta,g}
    \label{eq:order-rbar-non-adjusted}
  \end{gather}
\end{Theorem}
%%%%%%%%%%%%%%%%%%%%%%%%%%%%%%%%%%%%%%%%%%%%%%%%%%%%%%%%%%%%%%%%%%%

Note that $\epsilon = \gcd(c, N)$.
%
The width $w_\gamma$ of a cusp $\frac{a}{c}$ of $\Gamma_1(N)$ is given
by
\begin{gather}
  w_\gamma
  =
  \begin{cases}
    1,                   & \text{if $N=4$ and $\gcd(c,2)=2$},\\
    \frac{N}{\gcd(N,c)}, & \text{otherwise},
  \end{cases}
  \label{eq:width1}
\end{gather}
see Equation (2.49) in
\cite{Chen+Du+Zhao:FindingModularFunctionsRamanujan:2019} or Corollary
4 (2) of \cite{Cho+Koo+Park:ArithmeticRamanujanGoellnitzGordon:2009}
%
Since we always want an expansion in
$q^{1/w_\gamma}$, we adjust the following function for the special
case, so that if $A$ is the result of \code{matrixEtaOrder}$(N)$, then
$A \bar{r}$ gives the order of $g_{\bar{r}}(\gamma\tau)$ in terms of
$q^{1/w_\gamma}$.


Consider \eqref{eq:order-rbar-non-adjusted} at the cusps $\infty$ and
$0$ (which are $\Gamma_1(N)$-equivalent to $\frac{1}{N}$ and $1$,
respectively).
%
For the cusp $\infty$ we get $\lambda=\mu=1$ and $\epsilon=N$. Thus,
\begin{gather}
  \frac{1}{24}
  \left(
    \sum_{\divides{\delta}{N}} \delta r_\delta
    +
    12 \sum_{\substack{\divides{\delta}{N}\\0 < g < \frac{\delta}{2}}}
    \delta
    P_2\Bigl(\frac{g}{\delta}\Bigr)r_{\delta,g}
  \right).
  \label{eq:modularGamma1-sigmaInfinity}
\end{gather}

For the cusp $0$ we get $\lambda=\mu=1$ and $\epsilon=1$. Thus,
%
\begin{gather}
  \frac{1}{24}
  \left(
    \sum_{\divides{\delta}{N}} \frac{N}{\delta} r_\delta
    +
    2 \sum_{\substack{\divides{\delta}{N}\\0 < g < \frac{\delta}{2}}}
    \frac{N}{\delta} r_{\delta,g}
  \right)
  \label{eq:modularGamma1-sigma0}
\end{gather}
since $P_2(g)=P_2(0)=\frac{1}{6}$.


In other words, since these orders must be integers, we recover the
conditions \ref{eq:generalized-sigmaInfinity} and
\ref{eq:generalized-sigma0} from above.







%%%%%%%%%%%%%%%%%%%%%%%%%%%%%%%%%%%%%%%%%%%%%%%%%%%%%%%%%%%%%%%%%%%
\subsubsection{Transformation of Generalized Eta-Quotients under $\SL2Z$}
%%%%%%%%%%%%%%%%%%%%%%%%%%%%%%%%%%%%%%%%%%%%%%%%%%%%%%%%%%%%%%%%%%%

Let us take a generalized eta-quotient $g_{\bar{r}}$ of the form
\eqref{eq:g_rbar(tau)}. We aim to find the expansion of
$g_{\bar{r}}(\gamma\tau)$ for
$\gamma = \left(\begin{smallmatrix}a&b\\c&d\end{smallmatrix}\right)
\in \SL2Z$. Since the expansion of the pure eta-quotient
$g_r(\gamma\tau)$ (with $r=(r_{\delta,0})_{\divides{\delta}{N}}$ is
given by \eqref{eq:g_r(gamma*tau)}, we only need to combine the
transformations given by \eqref{eq:eta_g-h-N^[S](gamma*tau)}.

We followed the derivation in
Section~\ref{sec:eta_delta-transformation} and get (with the same
notation as in that section)
\begin{gather*}
  \delta\gamma\tau = \gamma_\delta \tau_\delta.
\end{gather*}

Let $\bar{r} = (r, \tilde{r})$ as defined in
Definition~\ref{def:rbar}. Since we have already a formula for a
transformation of a (pure) eta-quotient, we
define
\begin{align}
  \defineNotation[g-rtilde-tau]{g_{\tilde{r}}(\tau)}
  &=
  \prod_{\substack{\divides{\delta}{N}\\0<g<\frac{\delta}{2}}}
  \eta_{\delta,g}^{[R]}(\tau)^{r_{\delta,g}}
\end{align}
and thus split $g_{\bar{r}}(\tau)$ into a product of a (pure)
eta-quotient and a proper generalized eta-quotient, \ie,
\begin{align*}
  g_{\bar{r}}(\tau)
  &=
    g_r(\tau) \cdot g_{\tilde{r}}(\tau)
\end{align*}


Let us first find an expression for the transformation of
$\eta_{\delta,g}(\gamma\tau)$ which is implemented through the domain
\code{SymbolicGeneralizedEtaDeltaGamma}.

%%%%%%%%%%%%%%%%%%%%%%%%%%%%%%%%%%%%%%%%%%%%%%%%%%%%%%%%%%%%%%%%%%%
\begin{align}
  \defineNotation[eta-delta-g-gamma-tau-Robins]{\eta_{\delta,g}^{[R]}(\gamma\tau)}
  &=
    \eta_{g,0,\delta}^{[S]}(\delta\gamma\tau)
%  \notag\\
%%%%%%%%%%%%%%%%%%%%%%%%%%%%%%%%%%%%%%%%%%%%%%%%%%%%%%%%%%%%%%%%%%%
%  &
  =
    \eta_{g,0,\delta}^{[S]}(\gamma_\delta\tau_\delta)
%  \notag\\
%%%%%%%%%%%%%%%%%%%%%%%%%%%%%%%%%%%%%%%%%%%%%%%%%%%%%%%%%%%%%%%%%%%
%  &
   =
    \unityPower{\kappa_{g,0,\delta,\gamma_\delta}^{[S]}}\,
    \eta_{a_\delta g, b_\delta g,\delta}^{[S]}(\tau_\delta)
  \notag\\
%%%%%%%%%%%%%%%%%%%%%%%%%%%%%%%%%%%%%%%%%%%%%%%%%%%%%%%%%%%%%%%%%%%
  &=
    \unityPower{\kappa_{g,0,\delta,\gamma_\delta}^{[S]}}
    \cdot
    \alpha(a_\delta g, b_\delta g,\delta)
    \cdot
    q_\delta^{{\frac{1}{2} P_2\left(\frac{a_\delta g}{\delta}\right)}}
    \cdot
    \left(
    \zeta_\delta^{b_\delta g} q_\delta^{\left\{\frac{a_\delta g}{\delta}\right\}}
    ;
    q_\delta
    \right)
    \left(
    \zeta_\delta^{-{b_\delta g}} q_\delta^{1-\left\{\frac{a_\delta g}{\delta}\right\}}
    ;
    q_\delta
    \right)
  \notag\\
%%%%%%%%%%%%%%%%%%%%%%%%%%%%%%%%%%%%%%%%%%%%%%%%%%%%%%%%%%%%%%%%%%%
  \begin{split}
  &=
  \unityPower{\kappa_{g,0,\delta,\gamma_\delta}^{[S]}}
  \cdot
  \alpha(a_\delta g, b_\delta g,\delta)
  \cdot
  \unityPower{{\frac{v_\delta}{2} P_2\left(\frac{a_\delta g}{\delta}\right)}}
  \cdot
  q^{{\frac{u_\delta}{2} P_2\left(\frac{a_\delta g}{\delta}\right)}}
  \cdot
  \\
  &\qquad\qquad
  \cdot
  \left(
    \zeta_\delta^{b_\delta g} q_\delta^{\left\{\frac{a_\delta g}{\delta}\right\}}
    ;
    q_\delta
  \right)
  \left(
    \zeta_\delta^{-{b_\delta g}} q_\delta^{1-\left\{\frac{a_\delta g}{\delta}\right\}}
    ;
    q_\delta
  \right)
  \end{split}
  \notag\\
%%%%%%%%%%%%%%%%%%%%%%%%%%%%%%%%%%%%%%%%%%%%%%%%%%%%%%%%%%%%%%%%%%%
  \begin{split}
    &=
    \unityPower{
      \kappa_{g,0,\delta,\gamma_\delta}^{[S]}
      +
      \frac{v_\delta}{2} P_2\left(\frac{a_\delta g}{\delta}\right)
    }
    \cdot
    \alpha(a_\delta g, b_\delta g,\delta)
    \cdot
    \unityPower{\tau
      \frac{u_\delta}{2} P_2\left(\frac{a_\delta g}{\delta}\right)}
    \cdot
    \\
    &\qquad\qquad
    \cdot
    \left(
      \zeta_\delta^{b_\delta g} q_\delta^{\left\{\frac{a_\delta g}{\delta}\right\}}
      ;
      q_\delta
    \right)
    \left(
      \zeta_\delta^{-{b_\delta g}} q_\delta^{1-\left\{\frac{a_\delta g}{\delta}\right\}}
      ;
      q_\delta
    \right)
  \end{split}
  \label{eq:eta_delta-g^[R](gamma*tau)}
\end{align}
%%%%%%%%%%%%%%%%%%%%%%%%%%%%%%%%%%%%%%%%%%%%%%%%%%%%%%%%%%%%%%%%%%%

For ease of notation we use the following abbreviations:
%%%%%%%%%%%%%%%%%%%%%%%%%%%%%%%%%%%%%%%%%%%%%%%%%%%%%%%%%%%%%%%%%%%
\begin{alignat}{2}
  \defineNotation[sum-tilde]{\protect\widetilde{\protect\sum}}
  &=
  \sum_{\substack{\divides{\delta}{N}\\0<g<\frac{\delta}{2}}}
  &\qquad\text{and}\qquad
  \defineNotation[prod-tilde]{\protect\widetilde{\protect\prod}}
  &=
  \prod_{\substack{\divides{\delta}{N}\\0<g<\frac{\delta}{2}}}.
\end{alignat}
%%%%%%%%%%%%%%%%%%%%%%%%%%%%%%%%%%%%%%%%%%%%%%%%%%%%%%%%%%%%%%%%%%%

Thus, we have
%%%%%%%%%%%%%%%%%%%%%%%%%%%%%%%%%%%%%%%%%%%%%%%%%%%%%%%%%%%%%%%%%%%
\begin{align}
  \defineNotation[g-rtilde-gamma-tau]{g_{\tilde{r}}(\gamma\tau)}
  &=
    \widetilde{\prod}
    \eta_{\delta,g}^{[R]}(\gamma\tau)^{r_{\delta,g}}
  \notag\\
%%%%%%%%%%%%%%%%%%%%%%%%%%%%%%%%%%%%%%%%%%%%%%%%%%%%%%%%%%%%%%%%%%%
  \begin{split}
  &=
    \unityPower{
      \widetilde{\sum}
      r_{\delta,g}
      \left(
        \kappa_{g,0,\delta,\gamma_\delta}^{[S]}
      +
      \frac{v_\delta}{2} P_2\left(\frac{a_\delta g}{\delta}\right)
      \right)}
  \cdot
  \widetilde{\prod} \alpha(a_\delta g, b_\delta g,\delta)^{r_{\delta,g}}
  \cdot
  \\
  &\qquad\qquad
  \cdot
  \unityPower{\tau
    \widetilde{\sum}
    \frac{r_{\delta,g} u_\delta}{2} P_2\left(\frac{a_\delta g}{\delta}\right)}
  \cdot
  \widetilde{\prod}
  \left(
    \left(
      \zeta_\delta^{b_\delta g} q_\delta^{\left\{\frac{a_\delta g}{\delta}\right\}}
      ;
      q_\delta
  \right)
  \left(
    \zeta_\delta^{-{b_\delta g}} q_\delta^{1-\left\{\frac{a_\delta g}{\delta}\right\}}
    ;
    q_\delta
  \right)
  \right)^{r_{\delta,g}}
  \end{split}
  \label{eq:g_rtilde(gamma*tau)}
\end{align}
%%%%%%%%%%%%%%%%%%%%%%%%%%%%%%%%%%%%%%%%%%%%%%%%%%%%%%%%%%%%%%%%%%%
%
and, eventually,
%
%%%%%%%%%%%%%%%%%%%%%%%%%%%%%%%%%%%%%%%%%%%%%%%%%%%%%%%%%%%%%%%%%%%
\begin{align}
  \defineNotation[g-rbar-gamma-tau]{g_{\bar{r}}(\gamma\tau)}
  &=
    g_r(\gamma\tau) \cdot g_{\tilde{r}}(\gamma\tau)
  \notag\\
  \begin{split}
  &=
  (c \tau+d)^{\divisorsum{N} \frac{r_\delta}{2}}
  \cdot
  \divisorprod{N}
  \left(\frac{h_\delta}{\delta}\right)^{\!r_\delta/2}
  \cdot
  \unityPower{\divisorsum{N} \frac{r_\delta(v_\delta + \kappa_\delta)}{24}}
  \cdot
  q^{\divisorsum{N} \frac{r_\delta u_\delta}{24}}
  \cdot
  \divisorprod{N} (q_\delta; q_\delta)^{r_\delta}
  \cdot
  \\
  &\qquad\qquad
    \cdot
    \unityPower{
      \widetilde{\sum}
      r_{\delta,g}
      \left(
        \kappa_{g,0,\delta,\gamma_\delta}^{[S]}
      +
      \frac{v_\delta}{2} P_2\left(\frac{a_\delta g}{\delta}\right)
      \right)}
  \cdot
  \widetilde{\prod}
  \alpha(a_\delta g, b_\delta g,\delta)^{r_{\delta,g}}
  \cdot
  \\
  &\qquad\qquad
  \cdot
  \unityPower{\tau
    \widetilde{\sum}
    \frac{r_{\delta,g} u_\delta}{2} P_2\left(\frac{a_\delta g}{\delta}\right)}
  \cdot
  \widetilde{\prod}
  \left(
    \zeta_\delta^{b_\delta g} q_\delta^{\left\{\frac{a_\delta g}{\delta}\right\}}
    ;
    q_\delta
  \right)^{r_{\delta,g}}
  \left(
    \zeta_\delta^{-{b_\delta g}} q_\delta^{1-\left\{\frac{a_\delta g}{\delta}\right\}}
    ;
    q_\delta
  \right)^{r_{\delta,g}}
  \end{split}
  \label{eq:g_rbar(gamma*tau)}
\end{align}
%%%%%%%%%%%%%%%%%%%%%%%%%%%%%%%%%%%%%%%%%%%%%%%%%%%%%%%%%%%%%%%%%%%
where
\begin{align*}
  q_\delta = \unityPower{\tau_\delta}
  &=
    \unityPower{u_\delta\tau + v_\tau} = q^{u_\delta} \, \unityPower{v_\delta},
  \\
  \zeta_\delta
  &=
    \unityPower{\frac{1}{\delta}}
  \\
  \zeta_\delta^{b_\delta g} q_\delta^{\left\{\frac{a_\delta g}{\delta}\right\}}
  &=
    \unityPower{\frac{g(v_\delta a_\delta + b_\delta)}{\delta}}
    \unityPower{\tau u_\delta \left\{\frac{a_\delta g}{\delta}\right\}}
\end{align*}
and
$\kappa_{g,0,\delta,\gamma_\delta}$ is given by
equation~\eqref{eq:kappa_g-h-N-gamma-Schoeneberg} and
$g_r(\gamma\tau)$ is given by formula
\eqref{eq:g_r(gamma*tau)}.


Note that formula \eqref{eq:g_rbar(gamma*tau)} corresponds to (2.25)
in \cite{Chen+Du+Zhao:FindingModularFunctionsRamanujan:2019}.




%%%%%%%%%%%%%%%%%%%%%%%%%%%%%%%%%%%%%%%%%%%%%%%%%%%%%%%%%%%%%%%%%%%
\subsubsection{The function $g_{\bar{r},m,t}$ and its transformation
  under $\SL2Z$}
\label{sec:function-g_rbar-m-t}
%%%%%%%%%%%%%%%%%%%%%%%%%%%%%%%%%%%%%%%%%%%%%%%%%%%%%%%%%%%%%%%%%%%

In the following let $m$, $M$, $N$ be such that the conditions 1-10 of
Section~10 in
\cite{Chen+Du+Zhao:FindingModularFunctionsRamanujan:2019} are
fulfilled.
Furthermore let $\bar{r}\in\bar{R}(M)$.
%
We follow the derivation from Section~\ref{sec:function-g_r-m-t} and
define
%%%%%%%%%%%%%%%%%%%%%%%%%%%%%%%%%%%%%%%%%%%%%%%%%%%%%%%%%%%%%%%%%%%
\begin{align}
  \defineNotation[rtilde^{}]{\sumdelta{\tilde{r}}}
  &=
  12 \sum_{\substack{\divides{\delta}{M}\\0 < g < \frac{\delta}{2}}}
           \delta P_2\Bigl(\frac{g}{\delta}\Bigr) r_{\delta,g},
  \notag\\
  \defineNotation[rbar^{}]{\sumdelta{\bar{r}}}
  &= \sumdelta{r} + \sumdelta{\tilde{r}}
  =
  \divisorsum{M} \delta r_{\delta,0}
  + 12 \sum_{\substack{\divides{\delta}{M}\\0 < g < \frac{\delta}{2}}}
           \delta P_2\Bigl(\frac{g}{\delta}\Bigr) r_{\delta,g},
  \notag\\
  f(\tau)
  &=
  \sum_{n=0}^\infty a(n) q^n
    =
    \unityPower{\frac{- \tau \sumdelta{\bar{r}}}{24}} g_{\bar{r}}(\tau),
  \notag\\
  f_t(\tau)
  &=
    \unityPower{- t \tau} f(\tau)
    =
    \unityPower{\frac{- \tau (24t + \sumdelta{\bar{r}})}{24}} g_{\bar{r}}(\tau).
\end{align}
%%%%%%%%%%%%%%%%%%%%%%%%%%%%%%%%%%%%%%%%%%%%%%%%%%%%%%%%%%%%%%%%%%%

Recall that from \eqref{eq:U_m-f_t} that
%%%%%%%%%%%%%%%%%%%%%%%%%%%%%%%%%%%%%%%%%%%%%%%%%%%%%%%%%%%%%%%%%%%
\begin{align*}
  (U_mf_t)(\tau)
  &=
  \frac{1}{m} \sum_{\lambda=0}^{m-1}f_t\left(\frac{\tau+\lambda}{m}\right)
  =
    \sum_{n=0}^\infty a(mn+t) q^n.
\end{align*}
%%%%%%%%%%%%%%%%%%%%%%%%%%%%%%%%%%%%%%%%%%%%%%%%%%%%%%%%%%%%%%%%%%%

We can also evaluate $(U_mf_t)(\tau)$ in another way.
%
%%%%%%%%%%%%%%%%%%%%%%%%%%%%%%%%%%%%%%%%%%%%%%%%%%%%%%%%%%%%%%%%%%%
\begin{align*}
  (U_mf_t)(\tau)
  &=
  \frac{1}{m} \sum_{\lambda=0}^{m-1}f_t\left(\frac{\tau+\lambda}{m}\right)
%  \\
%  &
    =
    \frac{1}{m} \sum_{\lambda=0}^{m-1}
    \unityPower{-\frac{(\tau+\lambda) (24 t + \sumdelta{\bar{r}})}{24 m}}
    g_{\bar{r}}\!\left(\frac{\tau+\lambda}{m}\right)
  \\
% &=
% \frac{1}{m} \sum_{\lambda=0}^{m-1}
% \unityPower{-\frac{t\tau+t\lambda}{m} - \frac{\tau+\lambda}{24m} \sumdelta{r}}
% g_r\left(\frac{\tau+\lambda}{m}\right)\\
  &=
  \frac{1}{m}
  \unityPower{-\tau\frac{24t+\sumdelta{\bar{r}}}{24m}}
  \sum_{\lambda=0}^{m-1}
  \unityPower{-\frac{\lambda (24t+\sumdelta{\bar{r}})}{24m}}
  g_{\bar{r}}\!\left(\frac{\tau+\lambda}{m}\right)
  \\
  &=
  \frac{1}{m}
  \unityPower{-\tau\frac{24t+\sumdelta{\bar{r}}}{24m}}
  \sum_{\lambda=0}^{m-1} g_{\bar{r},m,t,\lambda}(\tau)
\end{align*}
%%%%%%%%%%%%%%%%%%%%%%%%%%%%%%%%%%%%%%%%%%%%%%%%%%%%%%%%%%%%%%%%%%%
with the abbreviation
%%%%%%%%%%%%%%%%%%%%%%%%%%%%%%%%%%%%%%%%%%%%%%%%%%%%%%%%%%%%%%%%%%%
\begin{align}
  \defineNotation[g-rbar-m-t-lambda-tau]{g_{\bar{r},m,t,\lambda}(\tau)}
  &:=
    \unityPower{-\frac{\lambda}{24m} (24t+\sumdelta{\bar{r}})}
    g_{\bar{r}}
      \!\left(
      \begin{pmatrix}
        1 & \lambda\\
        0 & m
      \end{pmatrix}
            \tau
      \right)
  \label{eq:g_rbar-m-t-lambda(tau)}
\end{align}
%%%%%%%%%%%%%%%%%%%%%%%%%%%%%%%%%%%%%%%%%%%%%%%%%%%%%%%%%%%%%%%%%%%

Eventually, we consider the function (similar to
\eqref{eq:g_r-m-t(tau)})
%%%%%%%%%%%%%%%%%%%%%%%%%%%%%%%%%%%%%%%%%%%%%%%%%%%%%%%%%%%%%%%%%%%
\begin{align}
  \defineNotation[g-rbar-m-t-tau]{g_{\bar{r},m,t}(\tau)}
  &:= \unityPower{\frac{\tau (24t+\sumdelta{\bar{r}})}{24m} } (U_mf_t)(\tau)
  =
  \frac{1}{m} \sum_{\lambda=0}^{m-1} g_{\bar{r},m,t,\lambda}(\tau)
  \label{eq:g_rbar-m-t(tau)}
\end{align}
%%%%%%%%%%%%%%%%%%%%%%%%%%%%%%%%%%%%%%%%%%%%%%%%%%%%%%%%%%%%%%%%%%%
and its transformations under $\SL2Z$.



Let us first investigate the transformation of
$g_{\bar{r},m,t,\lambda}(\tau)$ under some $\gamma\in\SL2Z$.
%
Since the prefactor in \eqref{eq:g_rbar-m-t-lambda(tau)} does not
depend on $\tau$, we essentially look for a transformation of the form
%%%%%%%%%%%%%%%%%%%%%%%%%%%%%%%%%%%%%%%%%%%%%%%%%%%%%%%%%%%%%%%%%%%
\begin{align}
  g_{\bar{r}}
  \left(
  \begin{pmatrix}
    1 & \lambda\\
    0 & m
  \end{pmatrix}
  \gamma
  \tau
  \right)
  &=
  g_r
  \left(
  \begin{pmatrix}
    1 & \lambda\\
    0 & m
  \end{pmatrix}
  \gamma
  \tau
  \right)
  \cdot
  g_{\tilde{r}}
  \left(
  \begin{pmatrix}
    1 & \lambda\\
    0 & m
  \end{pmatrix}
  \gamma
  \tau
  \right)
  \notag\\
  &=
  \divisorprod{M}
  \eta_\delta
  \left(
  \begin{pmatrix}
    1 & \lambda\\
    0 & m
  \end{pmatrix}
  \gamma
  \tau
  \right)^{r_\delta}
  \cdot
      \prod_{\substack{\divides{\delta}{M}\\0<g<\frac{\delta}{2}}}
  \eta_{\delta,g}^{[R]}
  \left(
  \begin{pmatrix}
    1 & \lambda\\
    0 & m
  \end{pmatrix}
  \gamma
  \tau
  \right)^{r_{\delta,g}},
  \notag\\
  &=
  \divisorprod{M}
  \eta(W\tau)^{r_\delta}
  \cdot
  \prod_{\substack{\divides{\delta}{M}\\0 < g < \frac{\delta}{2}}}
  \eta_{g,0,\delta}^{[S]}(W\tau)^{r_{\delta,g}},
  \notag\\
  &=
  \divisorprod{M}
  \eta(\gamma_{\delta,m,\lambda}\tau_{\delta,m,\lambda})^{r_\delta}
  \cdot
  \prod_{\substack{\divides{\delta}{M}\\0 < g < \frac{\delta}{2}}}
  \eta_{g,0,\delta}^{[S]}(\gamma_{\delta,m,\lambda}\tau_{\delta,m,\lambda})^{r_{\delta,g}},
\end{align}
%%%%%%%%%%%%%%%%%%%%%%%%%%%%%%%%%%%%%%%%%%%%%%%%%%%%%%%%%%%%%%%%%%%
where $W$, $\gamma_{\delta,m,\lambda}$, and $\tau_{\delta,m,\lambda}$
are defined in \eqref{eq:W_delta-m-lambda},
\eqref{eq:gamma_delta-m-lambda}, and
\eqref{eq:tau_delta-m-lambda}, respectively.


%%%%%%%%%%%%%%%%%%%%%%%%%%%%%%%%%%%%%%%%%%%%%%%%%%%%%%%%%%%%%%%%%%%
\begin{align}
  \defineNotation[g-rbar-m-t-lambda-gamma*tau]{g_{\bar{r},m,t,\lambda}(\gamma\tau)}
  &:=
  \unityPower{-\frac{\lambda}{24m} (24t+\sumdelta{\bar{r}})}
  \cdot
  \divisorprod{M}
  \eta(\gamma_{\delta,m,\lambda}\tau_{\delta,m,\lambda})^{r_\delta}
  \cdot
  \prod_{\substack{\divides{\delta}{M}\\0 < g < \frac{\delta}{2}}}
  \eta_{g,0,\delta}^{[S]}(\gamma_{\delta,m,\lambda}\tau_{\delta,m,\lambda})^{r_{\delta,g}}.
  \label{eq:g_rbar-m-t-lambda}
\end{align}
%%%%%%%%%%%%%%%%%%%%%%%%%%%%%%%%%%%%%%%%%%%%%%%%%%%%%%%%%%%%%%%%%%%










===============================================


%%%%%%%%%%%%%%%%%%%%%%%%%%%%%%%%%%%%%%%%%%%%%%%%%%%%%%%%%%%%%%%%%%%
\subsubsection{Transformation of $g_{\bar{r},m,t,\lambda}$ under $\SL2Z$}
%%%%%%%%%%%%%%%%%%%%%%%%%%%%%%%%%%%%%%%%%%%%%%%%%%%%%%%%%%%%%%%%%%%

Let $r$ and $\tilde{r}$ be defined as in Definition~\ref{def:rbar}.
Because of \eqref{eq:g_rbar-m-t-lambda(tau)} and
\eqref{eq:g_r-m-t-lambda(gamma*tau)}, we first establish a formula for
$g_{\tilde{r},m,0,\lambda}(\gamma\tau)$.
%%%%%%%%%%%%%%%%%%%%%%%%%%%%%%%%%%%%%%%%%%%%%%%%%%%%%%%%%%%%%%%%%%%
\begin{align}
  \defineNotation[g-rtilda-m-0-lambda]{g_{\tilde{r},m,0,\lambda}(\gamma\tau)}
  &=
  \unityPower{-\frac{\lambda \sumdelta{\tilde{r}}}{24m}}
  \cdot
  \prod_{\substack{\divides{\delta}{M}\\0 < g < \frac{\delta}{2}}}
  \eta_{\delta,g}
  \left(
  \begin{pmatrix}
    1 & \lambda\\
    0 & m
  \end{pmatrix}
  \gamma\tau
  \right)^{r_{\delta,g}},
  \notag\\
  &=
  \unityPower{-\frac{\lambda \sumdelta{\tilde{r}}}{24m}}
  \cdot
  \prod_{\substack{\divides{\delta}{M}\\0 < g < \frac{\delta}{2}}}
  \eta_{g,0,\delta}^{[S]}
  \left(
  \begin{pmatrix}
    \delta & \delta\lambda\\
    0 & m
  \end{pmatrix}
  \gamma\tau
  \right)^{r_{\delta,g}},
  \notag\\
  &=
  \unityPower{-\frac{\lambda \sumdelta{\tilde{r}}}{24m}}
  \cdot
  \prod_{\substack{\divides{\delta}{M}\\0 < g < \frac{\delta}{2}}}
  \eta_{g,0,\delta}^{[S]}
  \left(
  \gamma_{\delta,m,\lambda}\tau_{\delta,m,\lambda}
  \right)^{r_{\delta,g}},
  \notag\\
  &=
  \unityPower{-\frac{\lambda \sumdelta{\tilde{r}}}{24m}}
  \cdot
  \prod_{\substack{\divides{\delta}{M}\\0 < g < \frac{\delta}{2}}}
  \unityPower{\kappa_{g,0,\delta,\gamma_{\delta,m,\lambda}}^{[S]}} \,
  \eta_{a'g,b'g,\delta}^{[S]}
  \left(
  \tau_{\delta,m,\lambda}
  \right)^{r_{\delta,g}},
  \notag\\
  &=
  \unityPower{-\frac{\lambda \sumdelta{\tilde{r}}}{24m}}
  \cdot
  \unityPower{
    \sum_{\substack{\divides{\delta}{M}\\0 < g < \frac{\delta}{2}}}
    \kappa_{g,0,\delta,\gamma_{\delta,m,\lambda}}^{[S]}} \,
  \prod_{\substack{\divides{\delta}{M}\\0 < g < \frac{\delta}{2}}}
  \eta_{a'g,b'g,\delta}^{[S]}
  \left(
  \tau_{\delta,m,\lambda}
  \right)^{r_{\delta,g}},
  \label{eq:g_rtilde-m-0-lambda(gamma*tau)}
\end{align}
%%%%%%%%%%%%%%%%%%%%%%%%%%%%%%%%%%%%%%%%%%%%%%%%%%%%%%%%%%%%%%%%%%%
where $\gamma_{\delta,m,\lambda}$ and $\tau_{\delta,m,\lambda}$ are as
defined \eqref{eq:gamma_delta-m-lambda} and
\eqref{eq:tau_delta-m-lambda}, \ie,
$
\left(\begin{smallmatrix}
    \delta & \delta\lambda\\
    0 & m
\end{smallmatrix}\right)
\gamma\tau
=
\gamma_{\delta,m,\lambda}\,\tau_{\delta,m,\lambda}$.


In total we get
%%%%%%%%%%%%%%%%%%%%%%%%%%%%%%%%%%%%%%%%%%%%%%%%%%%%%%%%%%%%%%%%%%%
\begin{align}
  \defineNotation[g-rbar-m-r-lambda]{g_{\bar{r},m,t,\lambda}(\gamma\tau)}
  &=
  g_{r,m,t,\lambda}(\gamma\tau) \cdot g_{\tilde{r},m,0,\lambda}(\gamma\tau)
  \notag\\
  \begin{split}
  &=
    (c\tau+d)^{\divisorsum{M} \frac{r_\delta}{2}} \cdot
    \unityPower{-\frac{\lambda (24t+\sumdelta{\bar{r}})}{24m}
    + \divisorsum{M}\frac{r_\delta(v_{\delta,m,\lambda} +
    \kappa_{\gamma_{\delta,m,\lambda}})}{24}} \times\\
  & \qquad\times
    \divisorprod{M}
    \left(\frac{h_{\delta,m,\lambda}}{\delta}\right)^{\!\frac{r_\delta}{2}}
    \cdot
    \unityPower{\frac{\divisorsum{M} r_\delta u_{\delta,m,\lambda}}{24} \tau}
    \cdot
    \divisorprod{M}
    \prod_{n=1}^\infty (1-q_{\delta,m,\lambda}^n)^{r_\delta}
     \times\\
  & \qquad\times
    \unityPower{-\frac{\lambda \sumdelta{\tilde{r}}}{24m}}
    \cdot
    \unityPower{
      \sum_{\substack{\divides{\delta}{M}\\0 < g < \frac{\delta}{2}}}
      \kappa_{g,0,\delta,\gamma_{\delta,m,\lambda}}^{[S]}} \,
    \prod_{\substack{\divides{\delta}{M}\\0 < g < \frac{\delta}{2}}}
    \eta_{a'g,b'g,\delta}^{[S]}
    \left(
      \tau_{\delta,m,\lambda}
    \right)^{r_{\delta,g}}
  \end{split}
  \notag\\
  \begin{split}
  &=
    (c\tau+d)^{\divisorsum{M} \frac{r_\delta}{2}} \times\\
  & \qquad\times
    \unityPower{-\frac{\lambda (24t+\sumdelta{\bar{r}})}{24m}
      +
      \divisorsum{M}\frac{r_\delta(v_{\delta,m,\lambda} +
        \kappa_{\gamma_{\delta,m,\lambda}})}{24}
      +
      \sum_{\substack{\divides{\delta}{M}\\0 < g < \frac{\delta}{2}}}
      \kappa_{g,0,\delta,\gamma_{\delta,m,\lambda}}^{[S]}} \times\\
  & \qquad\times
    \divisorprod{M}
    \left(\frac{h_{\delta,m,\lambda}}{\delta}\right)^{\!\frac{r_\delta}{2}}
    \cdot
    \unityPower{\frac{\divisorsum{M} r_\delta u_{\delta,m,\lambda}}{24} \tau}
    \cdot
    \divisorprod{M}
    \prod_{n=1}^\infty (1-q_{\delta,m,\lambda}^n)^{r_\delta}
     \times\\
  & \qquad\times
    \prod_{\substack{\divides{\delta}{M}\\0 < g < \frac{\delta}{2}}}
    \eta_{a'g,b'g,\delta}^{[S]}
    \left(
      \tau_{\delta,m,\lambda}
    \right)^{r_{\delta,g}}
  \end{split}
  \label{eq:g_rbar-m-t-lambda(gamma*tau)}
\end{align}
%%%%%%%%%%%%%%%%%%%%%%%%%%%%%%%%%%%%%%%%%%%%%%%%%%%%%%%%%%%%%%%%%%%











Eventually, we get for the transformation of $g_{\bar{r},m,t}$ the
following.
%%%%%%%%%%%%%%%%%%%%%%%%%%%%%%%%%%%%%%%%%%%%%%%%%%%%%%%%%%%%%%%%%%%
\begin{align}
  g_{\bar{r},m,t}(\gamma\tau)
  &=
  \frac{1}{m} \sum_{\lambda=0}^{m-1} g_{\bar{r},m,t,\lambda}(\gamma\tau)
  \label{eq:g_rbar-m-t(gamma*tau)}
\end{align}
%%%%%%%%%%%%%%%%%%%%%%%%%%%%%%%%%%%%%%%%%%%%%%%%%%%%%%%%%%%%%%%%%%%






%%%%%%%%%%%%%%%%%%%%%%%%%%%%%%%%%%%%%%%%%%%%%%%%%%%%%%%%%%%%%%%%%%%
%%%%%%%%%%%%%%%%%%%%%%%%%%%%%%%%%%%%%%%%%%%%%%%%%%%%%%%%%%%%%%%%%%%
%%%%%%%%%%%%%%%%%%%%%%%%%%%%%%%%%%%%%%%%%%%%%%%%%%%%%%%%%%%%%%%%%%%
\bibliography{qeta}
\printindex
\end{document}
