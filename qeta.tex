\documentclass{article}
\usepackage{qeta}

\begin{document}
\title{The QEta Package}
\author{Ralf Hemmecke}
\maketitle
\begin{abstract}
  The QEta package is a collection of programs written in the
  FriCAS\footnote{FriCAS~1.3.2~\cite{FriCAS}} computer algebra system
  that allow to compute with Dedekind $\eta$-function and related
  $q$-series where $q=e^{2\pi i}$.
  \url{https://en.wikipedia.org/wiki/Dedekind_eta_function}
\end{abstract}

\section*{Overview of the files}

The QEta package contains implementation of the \algo{AB} algorithm
from \cite{Radu:RamanujanKolberg:2015} and the \algoSamba{} algorithm
from \cite{Hemmecke:DancingSambaRamanujan:2018}, in addition it
implements the algorithm from \cite{Hemmecke+Radu:EtaRelations:2018}
to compute all polynomial relations of Dedekind $\eta$-functions of a
certain level.

The underlying theory of the programs is described in the articles
\cite{Radu:RamanujanKolberg:2015},
\cite{Hemmecke:DancingSambaRamanujan:2018}, and
\cite{Hemmecke+Radu:EtaRelations:2018}.

This package requires a version of FriCAS that is compiled from at
least SVN revision 2328, \ie, where \GB{} computations do no longer
require variable names. I fact, the scripts usually use
\begin{verbatim}
)set output linear on
\end{verbatim}
which is coded in the file \PathName{1d.spad} in the
\code{master-hemmecke} branch of a clone of the FriCAS git repository.
\url{https://github.com/hemmecke/fricas/commits/master-hemmecke}
However, this \emph{linear} output form is not absolutely necessary.

The QEta package consists of following parts that are stored in the
respective \PathName{.spad} files. We mark the files that are only
there for historical reasons by a star. They are not really necessary
to compute the relation among $\eta$-functions.
\begin{description}
\item[qfunct] contains an implementation of (univariate) $q$-(Laurent)
  series for various well known functions like
  \begin{itemize}
  \item the $q$-Pochhammer symbol
    \url{https://en.wikipedia.org/wiki/Q-Pochhammer_symbol}
  \item the Euler function
    \url{https://en.wikipedia.org/wiki/Euler_function}, and
  \item the generating series for the partitions of natural numbers
    \url{https://en.wikipedia.org/wiki/Partition_(number_theory)#Generating_function}
  \end{itemize}

\item[qetaqmev] contains code for the construction of a basis for the
  monoid of all $\eta$-quotients of level $N$.

  The code is divided into two parts.
  \begin{itemize}
  \item Part 1 implements the construction as described in
    \cite{Radu:RamanujanKolberg:2015}.
  \item Part 2 implements the construction as described in
    \cite{Hemmecke+Radu:EtaRelations:2018}, \ie, it builds the
    respective matrices of an integer problem and then lets the
    program \algo{4ti2}\footnote{4ti2~1.6.7~\cite{4ti2}} solve that
    system.
  \end{itemize}

\item[qetadom] deals with a expansions of $\eta$-quotients at infinity
  in terms of $q$-series. These $\eta$-quotients generate an algebra
  over some ring $C$, the category of all such algebras is implemented
  via \code{QEtaAlgebra(C)}. A domain intended to record algebraic
  operations is given by \code{QEtaExtendedAlgebra}.

  Modular function that are $\eta$-quotients and have a pole (if any)
  only at infinity, can be represented by univariate Laurent series
  that have the property that they are zero of their order is
  positive. This domain is implemented via \code{Finite0Series}. In
  other words, in contrast to \spadtype{UnivariateLaurentSeries}, it
  can be checked in finite time whether an element of
  \code{Finite0Series} is zero or not. \code{Finite0Series(C, q, 0)}
  form a \code{QEtaAlgebra(C)}.

  An auxiliary domain is introduced that adds variable names so that a
  an element of a \code{DirectProduct} can be shown as a product
  of variables raised to some power.

\item[qetatool] contains a number of auxiliary packages, namely
  \begin{itemize}
  \item \code{QAuxiliaryToos} helps to create variables (symbols) with
    given indices and converts a polynomial from
    \spadtype{Polynomial}\code{(Q)} to \spadtype{Polynomial}\code{(Z)}
    by clearing denominators (where \code{Z==>}\spadtype{Integer} and
    \code{Q==>}\spadtype{Fraction}\code{(Z)}.
  \item \code{QFunctionToos} helps to create a series from another
    series by just picking a subsequence of coefficients. For example,
    from a series $\sum_{k=0}^\infty p(n)q^n$, we can form
    $\sum_{k=0}^\infty p(5n+4)q^n$.
  \item \code{Finite0SeriesFunctions2} helps to embed elements of
    \code{Finite0Series(C1, 'q, 0)} into \code{Finite0Series(C2, 'q,
      0)}.
  \item \code{PolynomialConversion} converts a polynomial from the
    generic \spadtype{Polynomial}\code{(C)} to a more specific polynomial ring of
    the form \code{PolynomialRing(C, E)} with a given domain of the
    exponents.
  \item \code{PolynomialEvaluation} evaluates the variables of a
    (multivariate) polynomial by given values in a structure that
    comes with the operations \code{+}, \code{*}, and \code{^}.
  \item \code{PolynomialTool} helps to extract from a number of
    polynomials only those whose variables start with a certain
    letter.
    %
    This package helps to compute $I\cap C[E]$ where
    $I \subset C[Y,E]$ and $Y=(Y_1,\ldots,Y_n$),
    $E=(E_1,\ldots, E_n)$.
  \item \code{ExtendedPolynomialReduction} implements a
    denominator-free polynomial reduction with respect to some basis
    that records the reduction steps.
  \item \code{QEtaGroebner} is a wrapper for the FriCAS
    \spadtype{GroebnerPackage}.
  \end{itemize}

\item[qetasamba] Implements the \algoSamba{} algorithm as
  \code{algebraBasis}.
  There are 2 variants of this algorithm that are implemented via the
  third argument of the \code{QEtaAlgebraBasis} package.

  The algorithm \algoSamba{} is a critical pair/completion algorithm.
  The variants basically vary in what the critical elements are.
  \begin{itemize}
  \item \code{QEtaComputation} Critical elements are pairs of basis
    elements that are recomputed after every extension of the basis.
    Since no products are computed, pairing basis elements is not
    considered too costly. Furthermore, basis elements that become
    reducible after an extension of the basis are kept in a separate
    critical element list.
  \item The second variant also uses \code{QEtaComputation}, but with
    \code{QEtaExtendedReduction} instead of \code{QEtaReduction} as
    its third argument.
  \end{itemize}
  Furthermore \code{QEtaReduction} implements the \emph{restricted
    reduction} as described in
  \cite{Hemmecke:DancingSambaRamanujan:2018}.

\item[qetaradu*] implements algorithms from
  \cite{Radu:RamanujanKolberg:2015}, in particular, the algorithms MC
  (membership check), MW (membership witness), MB (module basis), and
  AB (algebra basis).

\item[qetaicat] implements (as a default package of the category) the
  computation of polynomial relations among $\eta$ functions as
  described in \cite{Hemmecke+Radu:EtaRelations:2018}. The computation
  is done in various sub-steps building upon the known generators of
  the monoid of all $\eta$-quotients of a certain level, see file
  \PathName{qetaqmev.spad}.
  \begin{itemize}
  \item \code{etaQuotientIdealGenerators} computes generators of the
    ideal of all relations among $\eta$-quotients. The result
    corresponds to $H^{(M)}$ from
    \cite[Chapter~7]{Hemmecke+Radu:EtaRelations:2018}.
  \item \code{etaLaurentIdealGenerators} substitutes the $E_\delta$
    and $Y_\delta$ variables into the relations obtained from the
    previous step, this function returns $\chi'(H^L)\cup U$ as
    described in \cite[Chapter~7]{Hemmecke+Radu:EtaRelations:2018}
  \item \code{etaRelations} computes a \GB{} of the polynomials from
    the previous step and returns the polynomial that are only in the
    $E$ variables.
  \end{itemize}
  Variants of the computation are basically only in the way the
  generators are computed as implemented through the function
  \code{relationsIdealGenerators}.

\item[qetaih] is a package that implements the variant of the
  computation of $\eta$-relations as described in
  \cite{Hemmecke+Radu:EtaRelations:2018}.

\item[qeta3hdp*] implements a direct product domain with three block
  of degrevlex ordering. The helper domain
  \code{Split3HomogeneousDirectProduct} is used for the variants of
  computation of $\eta$-relations given through the files
  \PathName{qetair.spad} and \PathName{qetais.spad}.

\item[qetair*] implements a variant of the computation of
  $\eta$-relations. The algorithm \algo{AB} from
  \cite{Radu:RamanujanKolberg:2015} does not easily allow to express
  the resulting module basis elements in terms of the originally given
  algebra basis generators. Therefore, a \GB{} is computed to
  eliminate the variables corresponding to the module basis elements
  in order to obtain polynomials relations among the originally given
  $\eta$-quotients. In total there are then 3 \GB{} computations
  whereas via \PathName{qetaih.spad} only one \GB{} computation is
  necessary.

\item[qetais*] implements a variant of the computation of
  $\eta$-relations. This variant is somewhat in the middle between the
  implementations \PathName{qetaih.spad} and \PathName{qetair.spad} in
  the sense that the algorithm \algoSamba{} from
  \cite{Hemmecke:DancingSambaRamanujan:2018} is used instead of the
  algorithm \algo{AB} from \cite{Radu:RamanujanKolberg:2015}, but then
  it follows the computation as in \PathName{qetair.spad}.

\item[qetasomos] helps to compare the relation from Somos' table and
  the results from \code{etaRelations} by translating the notation
  back and forth.
\end{description}

\bibliography{qeta}
\end{document}
